\chapter{Aufgabe 5}

%----------------------------------------------------------------------------------------
%	FRÜHJAHR 2007
%----------------------------------------------------------------------------------------
\section{Frühjahr 2007}

\subsection{Aufgabe}
Es seien \( V = \{ (x_k)_{k \in \N} \mid x_k \in \R \} \) der reelle Vektorraum der reellen Folgen und \( \Phi \in \text{End}(V) \) der durch \( (x_k)_{k \in \N} \mapsto (x_{k+1})_{k \in \N} \) definierte Endomorphismus.
\\
Bestimmen Sie die Eigenwerte und die Eigenräume von \( \Phi \).

\subsection{Ansatz}
Stelle eine Bedingung für die Eigenwerte auf (hierbei kann vollständige Induktion benutzt werden). Finde eine Darstellung für die zu einem Eigenwert gehörigen Eigenvektoren (= Folge).

\subsection{Lösung}
Für \( c \in \R \) gilt: \( c \in \text{Spec}(\Phi) \), falls eine von der Nullfolge verschiedene Folge \( (x_k)_{k \in \N} \) existiert mit
\begin{equation*}
	\Phi((x_k)_{k \in \N}) = c(x_k)_{k \in \N}\text{.}
\end{equation*}
Das ist nach Definition von \( \Phi \) der Fall gdw \( \forall k \in \N: x_{k+1} = cx_k \). Durch Induktion erhalten wir
\begin{equation}
	\label{gl1}
	\forall k \in \N: x_{k+1} = c^kx_1
\end{equation}
Setzen wir \( x_1 = 1 \), so sehen wir, dass jedes \( c \in \R \) Eigenwert von \( \Phi \) ist.
\\
Ein zum Eigenwert \( c \in \R \) gehörender Eigenvektor ist die Folge \( (1, c, c_2, \dots) \).
\\
Andererseits folgt aus ~\ref{gl1}, dass jeder Eigenvektor zum Eigenwert \( c \) die Form \( (ac^{k-1})_{k \in \N}, a \in \R \) hat. Damit folgt für den zugehörigen Eigenraum:
\begin{equation*}
	E_c = \left[ (c^{k-1})_{k \in \N} \right] = [(1,c,c^2,\dots)]
\end{equation*}

\newpage

%----------------------------------------------------------------------------------------
%	HERBST 2007
%----------------------------------------------------------------------------------------
\section{Herbst 2007}

\subsection{Aufgabe}
In Abhängigkeit vom \( t \in \R \) sei folgende Matrix gegeben:
\begin{equation*}
	A_t \coloneqq \begin{pmatrix}
		t+4 & 0 & 5 & 1 \\
		0 & 2 & 0 & t \\
		1 & 0 & t & 1 \\
		0 & 0 & 0 & 2
	\end{pmatrix} \in \R^{4 \times 4}\text{.}
\end{equation*}
\begin{enumerate}
	\item Bestimmen Sie alle \( t \in \R \), für die \( A_t \) diagonalisierbar ist.
	\item Berechnen Sie eine reguläre Matrix \( S \in \R^{4 \times 4} \), für die \( S^{-1}A_0S \) diagonal ist. 
\end{enumerate}

\subsection{Ansatz}
\begin{enumerate}
	\item Damit \( A_t \) diagonalisierbar ist muss \( \text{CP}(A_t, X) \) in Linearfaktoren zerfallen und für jeden Eigenwert geometrische und algebraische Vielfachheit übereinstimmen.
	\item Betrachte die Matrix als Abbildung und transformiere sie durch eine Basiswechselmatrix \( S \) so, dass sie Diagonalform hat.
\end{enumerate}

\subsection{Lösung}
\begin{enumerate}
	\item Damit \( A_t \) diagonalisierbar ist, muss das charakteristische Polynom in Linearfaktoren zerfallen und für jeden Eigenwert die geometrische und algebraische Vielfachheit übereinstimmen.
	\begin{itemize}
	 	\item \emph{Charakteristisches Polynom}: Entwickelt man \( \text{det}(XI_4 - A_t) \) nach der letzten Zeile und dann nochmal nach der letzten Zeile, so erhält man
	 	\begin{equation*}
	 	 	\text{CP}(A_t, X) = (X-2)^2((X-t-5)(X-t)-5) = (X-2)^2(X^2-(2t+4)X + t^2 + 4t -5)
	 	 \end{equation*} 
	 	 Mit der Mitternachtsformel formt man um zu
	 	 \begin{equation*}
	 	 	\text{CP}(A_t, X) = (X - 2)^2(X - (t+5))(X - (t - 1))
	 	 \end{equation*}

	 	 \item \emph{Vielfachheiten}: Es muss nur die geometrische Vielfachheit von \( X=2 \) bestimmt werden, da für die anderen Nullstellen offensichtlich \( \mu_a = \mu_g \). \\
	 	 \( \mu_a(2) \geq 2 \), also muss auch \( \mu_g(2) \geq 2 \) sein, also \( \text{Rang}(A_t - 2I_4) \leq 4-2 = 2 \):
	 	 \begin{equation*}
	 	 	A_t-2I_4 = \begin{pmatrix}
	 	 		t+2 & 0 & 5 & 1 \\
	 	 		0 & 0 & 0 & t \\
	 	 		1 & 0 & t-2 & 1 \\
	 	 		0 & 0 & 0 & 0
	 	 	\end{pmatrix} \leadsto \begin{pmatrix}
	 	 		0 & 0 & 9-t^2 & -t-1 \\
	 	 		0 & 0 & 0 & t \\
	 	 		1 & 0 & t-2 & 1 \\
	 	 		0 & 0 & 0 & 0
	 	 	\end{pmatrix}
	 	 \end{equation*}
	 	 Es ist also \( \text{Rang}(A_t - 2I_4) = 2 \), wenn \( t \in \{ 0, 3, -3 \} \), also ist \( A_t \) höchstens für diese \( t \) diagonalisierbar.
	 	 \begin{itemize}
	 	 	\item \( t = 0 \): \( A_t \) ist diagonalisierbar, da hier \( \mu_g(2) = \mu_a(2) \).
	 	 	\item \( t = \pm 3 \): \( A_t \) ist nicht diagonalisierbar, da hier \( \mu_a(2)=3 \), aber \( \mu_g(2) = 2 \). 
	 	 \end{itemize}
	 \end{itemize}
	 Also ist \( A_t \) nur für \( t = 0 \) diagonalisierbar.

	 \item Es ist \( t = 0 \). Wir bestimmen Basen für alle Eigenräume:
	 \begin{itemize}
	 	\item \( \lambda = 2 \): Wir lösen \( (A_0-2I_4)v = 0 \) und erhalten \( b_1 \coloneqq \begin{pmatrix}
	 	 		0 & 1 & 0 & 0 
	 	 	\end{pmatrix}^\top, \ b_2 \coloneqq \begin{pmatrix}
	 	 		-7 & 0 & 1 & 9
	 	 	\end{pmatrix}^\top \).
	 	 \item \( \lambda = 5 \): Wir verfahren analog zu \( \lambda = 2 \) und erhalten \( b_3 \coloneqq \begin{pmatrix}
	 	 		5 & 0 & 1 & 0
	 	 	\end{pmatrix}^\top \).
	 	 \item \( \lambda = -1 \): Wir verfahren analog zu \( \lambda = 2 \) und erhalten \( b_4 \coloneqq \begin{pmatrix}
	 	 		1 & 0 & -1 & 0
	 	 	\end{pmatrix}^\top \).
	 \end{itemize}
	 Daher ist \( \{ b_1, b_2, b_3, b_4 \} \) eine Basis aus Eigenvektoren. Die Matrix \( S \coloneqq (b_1, b_2, b_3, b_4) \) ist regulär und erfüllt die geforderte Bedingung: \( S^{-1}A_0S = \text{diag}(2, 2, 5, -1) \).
\end{enumerate}


%----------------------------------------------------------------------------------------
%	HERBST 2010
%----------------------------------------------------------------------------------------
\section{Herbst 2010}

\subsection{Aufgabe}
Gegeben seien die zwei reellen Matrizen
\begin{equation*}
	A = \begin{pmatrix}
		1 & 1 & -1 \\
		-2 & 2 & 1 \\
		-2 & 1 & 2
	\end{pmatrix}, \qquad B = \begin{pmatrix}
		1 & 0 & 0 \\
		-2 & 1 & 2 \\
		-2 & 0 & 3
	\end{pmatrix}\text{.}
\end{equation*}
\begin{enumerate}
	\item Zeigen Sie, dass \( A \) und \( B \) dasselbe charakteristische Polynom haben.
	\item Welche der Matrizen sind diagonalisierbar?
	\item Bestimmen Sie für eine der Matrizen eine Basis des \( \R^3 \), die aus Eigenvektoren besteht.
\end{enumerate}

\subsection{Ansatz}
\begin{enumerate}
	\item Bestimme \( \text{CP}_A(X) \) und \( \text{CP}_B(X) \).
	\item Eine Matrix ist dann diagonalisierbar, wenn das charakteristische Polynom in Linearfaktoren zerfällt und für jeden Eigenwert algebraische und geometrische Vielfachheit übereinstimmen.
	\item Löse je Eigenwert \( \lambda \in \text{Spec}(M) \): \( \text{Kern}(M - \lambda I_3) \) (\( M \in \{ A, B \} \)). 
\end{enumerate}

\subsection{Lösung}
\begin{enumerate}
	\item Wir bestimmen jeweils das charakteristische Polynom:
	\begin{itemize}
	 	\item \( A \): \( \text{CP}_A(X) = \text{det}\begin{pmatrix}
	 		X-1 & -1 & 1 \\
	 		2 & X-2 & -1 \\
	 		2 & -1 & X-2
	 	\end{pmatrix} = (X-1)^2(X-3) \). 

	 	\item \( B \): \( \text{CP}_B(X) = \text{det}\begin{pmatrix}
	 		X-1 & 0 & 0 \\
	 		2 & X-1 & -2 \\
	 		2 & 0 & X-3
	 	\end{pmatrix} = (X-1)^2(X-3) \).
	 \end{itemize} 

	 \item Es ist \( \text{Spec}(M) = \{ 1, 3 \} \), \( \mu_a(1) = 2 \), \( \mu_a(3) = 1 \) für \( M \in \{ A, B \} \).
	 \begin{itemize}
	 	\item \( A \): Der Eigenraum zum Eigenwert \( 1 \) ist:
	 	\begin{equation*}
	 	 	\text{Kern}(A-I_3) = \begin{pmatrix}
	 	 		0 & 1 & -1 \\
	 	 		-2 & 1 & 1 \\
	 	 		-2 & 1 & 1
	 	 	\end{pmatrix}\text{.}
	 	 \end{equation*}
	 	 Diese Matrix hat Rang \( 2 \), also ist \( \text{dim}(\text{Eig}(A, 1)) = 1 \), also \( \mu_a(1) \neq \mu_g(1) \). Also ist \( A \) nicht diagonalisierbar.

	 	 \item \( B \):
	 	 \begin{itemize}
	 	  	\item \( \lambda = 1 \): \( \text{Kern}(B - I_3) = \text{Kern}\begin{pmatrix}
	 	  		0 & 0 & 0 \\
	 	  		-2 & 0 & 2 \\
	 	  		-2 & 0 & 2
	 	  	\end{pmatrix} = \langle \begin{pmatrix}
	 	  		0 \\ 1 \\ 0
	 	  	\end{pmatrix}, \begin{pmatrix}
	 	  		1 \\ 0 \\ 1
	 	  	\end{pmatrix} \rangle \) 

	 	  	\item \( \lambda = 3 \): \( \text{Kern}(B - 3I_3) = \text{Kern}\begin{pmatrix}
	 	  		-2 & 0 & 0 \\
	 	  		-2 & -2 & 2 \\
	 	  		-2 & 0 & 0
	 	  	\end{pmatrix} = \langle \begin{pmatrix}
	 	  		0 \\ 1 \\ 1
	 	  	\end{pmatrix} \rangle \)
	 	  \end{itemize} 
	 	  Also ist \( B \) diagonalisierbar.
	 \end{itemize}

	 \item Nur für diagonalisierbare Matrizen kann eine solche Basis besimmt werden, also nur für \( B \).
	 	\\
	 	Eine solche Basis ist nach Rechnung oben
	 	\begin{equation*}
	 		\left \{ \begin{pmatrix}
	 			0 \\ 1 \\ 0
	 		\end{pmatrix}, \begin{pmatrix}
	 			1 \\ 0 \\ 1
	 		\end{pmatrix}, \begin{pmatrix}
	 			0 \\ 1 \\ 1
	 		\end{pmatrix} \right \}\text{.}
	 	\end{equation*}
\end{enumerate}

\newpage

%----------------------------------------------------------------------------------------
%	FRÜHJAHR 2013
%----------------------------------------------------------------------------------------
\section{Frühjahr 2013}

\subsection{Aufgabe}
Im Vektorraum \( \R^4 \) sei für \( t \in \R \) ein Endomorphismus \( \Phi_t: \R^4 \ni x \mapsto A_tx \in \R^4 \) gegeben mit
\begin{equation*}
	A_t = \begin{pmatrix}
		2+t & 4 & 2+t & 2+t \\
		t-2 & 0 & -6+t & -2-t \\
		-t+2 & -4 & -t+2 & -2-t \\
		0 & 0 & 0 & 2t
	\end{pmatrix}\text{.}
\end{equation*}
\begin{enumerate}
	\item Berechnen Sie alle Eigenwerte von \( \Phi_t \).
	\item Bestimmen Sie alle \( t \in \R \), für die \( \Phi_t \) diagonalisierbar ist.
	\item Berechnen Sie für \( t = 2 \) eine Basis von \( \R^4 \) aus Eigenvektoren von \( \Phi_t \).  
\end{enumerate}

\subsection{Ansatz}
\begin{enumerate}
	\item Bestimme das charakteristische Polynom von \( \Phi_t \).
	\item Bestimme alle \( t \in \R \), für die \( \text{CP}_{\Phi_t}(X) \) in Linearfaktoren zerfällt und für jeden Eigenwert geometrische und algebraische Vielfachheit übereinstimmen.
	\item Nutze hierfür die Basisvektoren der Eigenräume aus dem zweiten Teil. 
\end{enumerate}

\subsection{Lösung}
\begin{enumerate}
	\item Wir bestimmen \( \text{CP}_{\Phi_t}(X) \):
	\begin{equation*}
	 	\text{CP}_{\Phi_t}(X) = \text{det}(A_t - XI_4) = (X - 2t)(X - 4)^2(X+4)
	 \end{equation*} 
	 Es ist also \( \text{Spec}(\Phi_t) = \{ 2t, -4, 4 \} \).

	 \item Das charakteristische Polynom zerfällt offensichtlich in Linearfaktoren. Wir betrachten die Vielfachheiten vom Eigenwert \( 4 \): 
	 \begin{equation*}
	 	\text{Rang}(A_t - 4I_4) = \text{Rang}\begin{pmatrix}
	 		-2+t & 4 & 2+t & 2+t \\
	 		0 & -8 & -8 & -4-2t \\
	 		0 & 0 & 0 & 0 \\
	 		0 & 0 & 0 & -4+2t
	 	\end{pmatrix} = \begin{cases}
	 		1, \quad \text{falls } t = 2 \\
	 		0, \quad \text{sonst}
	 	\end{cases}
	 \end{equation*}
	 \begin{itemize}
	 	\item \( t \neq 2 \): \( \text{dim}(\text{Eig}(A_t, 4)) = 4 - 3 = 1 < 2 \), also ist \( A_t \) für \( t \neq 2 \) nicht diagonalisierbar.
	 	\item \( t = 2 \): \( \text{dim}(\text{Eig}(A_t, 4)) = 4 - 1 = 3 = \mu_a(4) \quad \checkmark \)  
	 \end{itemize}
	 Es ist \( \text{dim}(\text{Eig}(A_2, -4)) = 1 = \mu_a(-4) \). Also ist \( A_2 \) diagonalisierbar.

	 \item Wir berechnen zu den beiden Eigenwerten Basen der zugehörigen Eigenräume:
	 	\begin{itemize}
	 		\item \( X = 4 \): \( \text{Eig}(A_2, 4) = \text{Kern}(A_2 - 4 I_4) = \text{Kern}\begin{pmatrix}
	 			0 & 0 & 0 & 0 \\
	 			0 & 1 & 1 & 1 \\
	 			0 & 0 & 0 & 0 \\
	 			0 & 0 & 0 & 0
	 		\end{pmatrix} = \left [  \begin{pmatrix}
	 			1 \\ 0 \\ 0 \\ 0 
	 		\end{pmatrix}, \begin{pmatrix}
	 			0 \\ 1 \\ -1 \\ 0 
	 		\end{pmatrix}, \begin{pmatrix}
	 			0 \\ 1 \\ 0 \\ -1
	 		\end{pmatrix} \right ] \)

	 		\item \( X = -4 \): \( \text{Eig}(A_2, -4) = \text{Kern}(A_2 + 4I_4) = \text{Kern}\begin{pmatrix}
	 			1 & 0 & 1 & 0 \\
	 			0 & 1 & -1 & 0 \\
	 			0 & 0 & 0 & 0 \\
	 			0 & 0 & 0 & 1
	 		\end{pmatrix} = \left [ \begin{pmatrix}
	 			1 \\ -1 \\ -1 \\ 0
	 		\end{pmatrix} \right ] \)
	 	\end{itemize}
	 	Eine Basis aus Eigenvektoren von \( A_2 \) ist also
	 	\begin{equation*}
	 		\left\{ \begin{pmatrix}
	 			1 \\ 0 \\ 0 \\ 0
	 		\end{pmatrix}, \begin{pmatrix}
	 			0 \\ 1 \\ -1 \\ 0
	 		\end{pmatrix}, \begin{pmatrix}
	 			0 \\ 1 \\ 0 \\ -1
	 		\end{pmatrix}, \begin{pmatrix}
	 			1 \\ -1 \\ -1 \\ 0
	 		\end{pmatrix} \right\}\text{.}
	 	\end{equation*}
\end{enumerate}

\newpage

%----------------------------------------------------------------------------------------
%	HERBST 2013
%----------------------------------------------------------------------------------------
\section{Herbst 2013}

\subsection{Aufgabe}
Sei \( n \in \N \) und \( A \in \C^{n \times n} \) eine Matrix, für die es ein \( k \in \N \) gibt mit \( A_k = I_n \) gibt.
\begin{enumerate}
	\item Geben Sie ein von Null verschiedenes annullierendes Polynom für \( A \) an.
	\item Weisen Sie nach, dass \( A \) diagonalisierbar ist.
\end{enumerate}
\begin{remark}
	\emph{Hinweis}: \\
	Jedes nicht konstante komplexe Polynom ist ein Produkt von Linearfaktoren.
\end{remark}

\subsection{Ansatz}
\begin{enumerate}
	\item Finde ein Polynom \( p \in \C[X] \), sodass \( p(A) = 0 \).
	\item \( A \) ist diagonalisierbar, wenn das Minimalpolynom in einfache Linearfaktoren zerfällt. Zeige, dass das der Fall ist, indem du eine Nullstelle \( z \) annimmst, das annullierende Polynom aus dem ersten Teil durch \( (X-z) \) teilst und zeigst, dass bei Einsetzen von \( z \) in das resultierende Polynom ein Ergebnis \( \neq 0 \) vorliegt.
\end{enumerate}

\subsection{Lösung}
\begin{enumerate}
	\item Wegen \( A^k = I_n \) ist \( X^k-1 \) ein annullierendes Polynom vom Grad \( k > 0 \), also \( \neq 0 \).
	\item \( \text{MP}_A(\lambda) \) teilt \( X^k-1 \) und zerfällt in Linearfaktoren. Damit \( A \) diagonalisierbar ist darf also keine mehrfache Nullstelle in \( \text{MP}_A(\lambda) \) vorkommen.
		\\
		Sei \( z \) Nullstelle des Minimalpolynoms (\( z^k = 1 \leadsto z \neq 0 \)). Wenn \( z \) mehrfache Nullstelle von \( \text{MP}_A(\lambda) \) wäre, dann auch von \( X^k-1 \). Wir teilen \( X^k-1 \) durch \( X-z \): 
		\begin{equation*}
			(X^k-1):(X-z) = X^{k-1} + zX^{k-2} + \cdots + z^{k-2}X + z^{k-1} = \sum_{i = 0}^{k-1} z^iX^{k-1-i} 
		\end{equation*}
		Setzt man rechts \( z \) ein, so erhält man \( k*z^{k-1} \neq 0 \), also ist \( z \) einfache Nullstelle von \( \text{MP}_A(\lambda) \).
\end{enumerate}

\newpage

%----------------------------------------------------------------------------------------
%	FRÜHJAHR 2014
%----------------------------------------------------------------------------------------
\section{Frühjahr 2014}

\subsection{Aufgabe}
In Abhängigkeit von \( b \in \R \) sei folgende reelle Matrix gegeben:
\begin{equation*}
	A_b = \begin{pmatrix}
		b & b -1 & 0 \\
		-3 & -2 & 0 \\
		b+1 & b-1 & -1
	\end{pmatrix}
\end{equation*}
\begin{enumerate}
	\item Für welche \( b \in \R \) ist \( A_b \) diagonalisierbar?
	\item Bestimmen Sie für \( b = 2 \) eine invertierbare reelle Matrix \( S \) und eine Diagonalmatrix \( D \), sodass \( D = S^{-1}A_2S \). 
\end{enumerate}

\subsection{Ansatz}
\begin{enumerate}
	\item Eine Matrix ist genau dann diagonalisierbar, wenn ihr charakteristisches Polynom in Linearfaktoren zerfällt und für jeden Eigenwert geometrische und algebraische Vielfachheit übereinstimmen.
	\item Konstruiere \( S \) aus Basisvektoren der Eigenräume der Eigenwerte von \( A_2 \). 
\end{enumerate}

\subsection{Lösung}
\begin{enumerate}
	\item Zuerst bestimmen wir das charakteristische Polynom von \( A_b \):
	\begin{equation*}
	 	\text{CP}_{A_b}(\lambda) = -\text{det}(A_b - \lambda I_3) = (\lambda + 1)(\lambda - 1)(\lambda - (b-3))
	 \end{equation*} 
	 \begin{itemize}
	 	\item \( b \not \in \{ 2, 4 \} \): \( A_b \) hat drei verschiedene Eigenwerte \( \leadsto \) \( A_b \) diagonalisierbar.
	 	\item \( b = 2 \): \( \mu_a(-1) = 2 \). Wir berechnen \( \mu_g(-1) \):
	 	\begin{equation*}
	 	 	\text{Kern}(A_2 + I_3) = \text{Kern}\begin{pmatrix}
	 	 		3 & 1 & 0 \\
	 	 		-3 & -1 & 0 \\
	 	 		3 & 1 & 0 \\
	 	 	\end{pmatrix} = \langle \begin{pmatrix}
	 	 		0 \\ 0 \\ 1
	 	 	\end{pmatrix}, \begin{pmatrix}
	 	 		1 \\ -3 \\ 0
	 	 	\end{pmatrix} \rangle
	 	 \end{equation*} 
	 	 Also ist hier \( \mu_a(-1) = \mu_g(-1) \) und \( A_2 \) somit diagonalieierbar.

	 	 \item \( b = 4 \): \( \mu_a(1) = 2 \). Wir berechnen \( \mu_g(1) \):
	 	 \begin{equation*}
	 	 	\text{Kern}(A_4-I_3) = \text{Kern}\begin{pmatrix}
	 	 		1 & 0 & -1 \\
	 	 		0 & 1 & 1 \\
	 	 		0 & 0 & 0 
	 	 	\end{pmatrix} = \langle \begin{pmatrix}
	 	 		-1 \\ 1 \\ -1
	 	 	\end{pmatrix} \rangle
	 	 \end{equation*}
	 	 Da \( \mu_g(1) = 1 \) ist \( A_4 \) nicht diagonalisierbar.
	 \end{itemize}

	 \item Wir können \( S \) und \( D \) direkt angeben, da wir eben schon alle Basen der Eigenräume von \( A_2 \) berechnet haben:
	 \begin{equation*}
	 	S = \begin{pmatrix}
	 		-1 & 0 & 1 \\
	 		1 & 0 & -3 \\
	 		-1 & 1 & 0
	 	\end{pmatrix}, \quad D = \begin{pmatrix}
	 		1 & 0 & 0 \\
	 		0 & -1 & 0 \\
	 		0 & 0 & -1
	 	\end{pmatrix}
	 \end{equation*}
\end{enumerate}

\newpage

%----------------------------------------------------------------------------------------
%	HERBST 2014
%----------------------------------------------------------------------------------------
\section{Herbst 2014}

\subsection{Aufgabe}
Für \( s \in \R \) sei folgende reelle Matrix gegeben:
\begin{equation*}
	A_s = \begin{pmatrix}
		s & 0 & 0 & s-2 \\
		2 & s+1 & -1 & 0 \\
		2s & s(s+1) & -s & 3 \\
		0 & 0 & 0 & s
	\end{pmatrix}
\end{equation*}
\begin{enumerate}
	\item Berechne \( \spec(A_s) \) für alle \( s \in \R \).
	\item Für welche \( s \in \R \) ist \( A_s \) diagonalisierbar?
	\item Bestimme eine \( \R^4 \)-Basis aus \( A_2 \)-Eigenvektoren. 
\end{enumerate}

\subsection{Ansatz}
\begin{enumerate}
	\item Bestimme das charakteristische Polynom von \( A_s \) und lese die Eigenwerte ab.
	\item Eine Matrix ist genau dann diagonalisierbar, wenn das charaktieristische Polynom in Linearfaktoren zerfällt und für jeden Eigenwert algebraische und geometrische Vielfachheit übereinstimmen.
	\item Konstruiere die Basis aus Basisvektoren der Eigenräume. 
\end{enumerate}

\subsection{Lösung}
\begin{enumerate}
	\item Wir berechnen \( \cp_{A_s}(\lambda) \):
	\begin{equation*}
	 	\cp_{A_s}(\lambda) = \det \begin{pmatrix}
	 		s - \lambda & 0 & 0 & s-2 \\
	 		2 & 2+1-\lambda & -1 & 0 \\
	 		2s & s(s+1) & -s-\lambda & 3 \\
	 		0 & 0 & 0 & s-\lambda
	 	\end{pmatrix} = X(X-1)(X-s)^2
	 \end{equation*} 
	 Also ist \( \spec(A_s) = \{ 0, 1, s \} \).

	 \item Offensichtlich zerfällt \( \cp_{A_s}(\lambda) \) in Linearfaktoren. Es bleibt zu bestimmen, für welche \( s \in \R \) die algebraischen und geometrischen Vielfachheiten aller Eigenwerte gleich sind.
	 \begin{itemize}
	 	\item \( \lambda = s \): Wir bestimmen \( \dim(\kr(A_s - sI_4)) \):  
 		\begin{equation*}
 			\kr(A_s - sI_4) = \begin{pmatrix}
 				0 & 0 & 0 & s-2 \\
 				2 & 1 & -1 & 0 \\
 				2s & s(s+1) & -2s & 3 \\
 				0 & 0 & 0 & 0
 			\end{pmatrix} \leadsto \begin{pmatrix}
 				1 & \tfrac{1}{2} & -\tfrac{1}{2} & 0 \\
 				0 & s^2 & -s & 3 \\
 				0 & 0 & 0 & 0 \\
 				0 & 0 & 0 & s-2
 			\end{pmatrix}
 		\end{equation*}
 		Es ist also \( \dim(\eig(A_s, s)) = \begin{cases}
 			2, \quad \text{falls } s \in \{ 0, 2 \} \\
 			1, \quad \text{sonst}
 		\end{cases} \). \\ Somit ist \( A_s \) höchstens für \( s \in \{ 0, 2 \} \) diagonalisierbar.
	 \end{itemize}
	 Ist \( s = 0 \), so ist \( \mu_a(s) = 3 \) und \( A_s \) somit nicht diagonalisierbar. \\
	 Ist \( s = 2 \), so ist \( \mu_a(s) = \mu_g(s) \). Für die anderen Eigenwerte gilt das sowieso, also ist \( A_s \) genau für \( s = 2 \) diagonalisierbar.

	 \item Wir berechnen eine Basis für jeden Eigenraum \( E_\lambda = \kr(A_2 - \lambda I_4) \), indem wir jeweils \( (A_2 - \lambda I_4)v = 0 \) lösen. Wir erhalten die Vektoren
	 \begin{equation*}
	 	\begin{pmatrix}
	 		1 \\ 2 \\ 4 \\ 0
	 	\end{pmatrix}, \begin{pmatrix}
	 		3 \\ -6 \\ 0 \\ 8
	 	\end{pmatrix}, \begin{pmatrix}
	 		0 \\ 1 \\ 2 \\ 0
	 	\end{pmatrix}, \begin{pmatrix}
	 		0 \\ 1 \\ 3 \\ 0
	 	\end{pmatrix}\text{.}
	 \end{equation*}
	 Diese bilden eine \( \R^4 \)-Basis aus \( A_2 \)-Eigenvektoren.
\end{enumerate}

\newpage

%----------------------------------------------------------------------------------------
%	FRÜHJAHR 2015
%----------------------------------------------------------------------------------------
\section{Frühjahr 2015}

\subsection{Aufgabe}
Für \( s \in \R \) sei folgende reelle Matrix gegeben:
\begin{equation*}
	A_s = \begin{pmatrix}
		s-1 & 0 & s \\
		s-2 & 1 & s \\
		3 & 0 & 2
	\end{pmatrix}
\end{equation*}
\begin{enumerate}
	\item Bestimme \( \spec(A_s) \) für alle \( s \in \R \).
	\item Bestimme alle \( s \in \R \), für die \( A_s \) diagonalisierbar ist.
	\item Bestimme für \( s = -1 \) eine \( \R^3 \)-Basis aus \( A_{-1} \)-Eigenvektoren. 
\end{enumerate}

\subsection{Ansatz}
\begin{enumerate}
	\item Bestimme das charakteristische Polynom von \( A_s \) und lese die Eigenwerte ab.
	\item Eine Matrix ist genau dann diagonalisierbar, wenn ihr charakteristisches Polynom in Linearfaktoren zerfällt und für jeden Eigenwert algebraische und geometrische Vielfachheit übereinstimmen.
	\item Eine solche Basis besteht aus den Basisvektoren der Eigenräume. 
\end{enumerate}

\subsection{Lösung}
\begin{enumerate}
	\item Wir bestimmen das charakteristische Polynom von \( A_s \):
	\begin{equation*}
	 	\cp_{A_s}(\lambda) = \det \begin{pmatrix}
	 		s - 1 - \lambda & 0 & s \\
	 		s-2 & 1 - \lambda & s \\
	 		3 & 0 & 2-\lambda
	 	\end{pmatrix} = (1-\lambda)(-1-\lambda)((2+s) - \lambda)
	 \end{equation*} 
	 Es ist also \( \spec(A_s) = \{ 1, -1, 2+s \} \).

	 \item Für \( s \in \R \setminus \{ -1 , -3 \} \) hat \( A_s \) drei verschiedene Eigenwerte und ist also diagonalisierbar.
	 \begin{itemize}
	 	\item \( s = -1 \): Hier ist \( \mu_a(1) = 2 \). Wir bestimmen \( \mu_g(1) \):
	 	\begin{equation*}
	 	 	\mu_g(1) = \dim(\kr(A_{-1} - I_3)) = \dim \begin{pmatrix}
	 	 		1 & 0 & \tfrac{1}{3} \\
	 	 		0 & 0 & 0 \\
	 	 		0 & 0 & 0 
	 	 	\end{pmatrix} = 2\text{.}
	 	 \end{equation*} 
	 	 Also ist \( A_{-1} \) diagonalisierbar.

	 	 \item \( s = -3 \): Hier ist \( \mu_a(-1) = 2 \). Wir bestimmen \( \mu_g(-1) \):
	 	 \begin{equation*}
	 	 	\mu_g(-1) = \dim(\kr(A_{-3} + I_3)) = \dim \begin{pmatrix}
	 	 		1 & 0 & 1 \\
	 	 		0 & 2 & 2 \\
	 	 		0 & 0 & 0 
	 	 	\end{pmatrix} = 1\text{.}
	 	 \end{equation*}
	 	 Also ist \( A_{-3} \) nicht diagonalisierbar.
	 \end{itemize}
	 Insgesamt ist \( A_s \) für \( s \in \R \setminus \{ -3 \} \) diagonalisierbar.

	 \item Eine solche Basis ist die Basis aus den Basisvektoren der Eigenräume, die man aus den beiden obigen Matrizen erhält:
	 \begin{equation*}
	 	\langle \begin{pmatrix}
	 		0 \\ 1 \\ 0
	 	\end{pmatrix}, \begin{pmatrix}
	 		1 \\ 0 \\ -3
	 	\end{pmatrix}, \begin{pmatrix}
	 		1 \\ 1 \\ -1
	 	\end{pmatrix} \rangle = \R^3
	 \end{equation*}
\end{enumerate}

\newpage

%----------------------------------------------------------------------------------------
%	HERBST 2015
%----------------------------------------------------------------------------------------
\section{Herbst 2015}

\subsection{Aufgabe}
In Abhänigkeit von \( t \in \R \) sei folgende reelle Matrix gegeben:
\begin{equation*}
 A_t = \begin{pmatrix}
 	1+t & 1 & 1-t \\
 	-1 & t-1 & t-1 \\
 	1 & 2-2t & 2-2t
 \end{pmatrix}
\end{equation*}
\begin{enumerate}
	\item Für welche \( t \in \R \) ist \( A_t \) diagonalisierbar?
	\item Gebe für \( t = \tfrac{1}{2} \) eine \( \R^3 \)-Basis aus \( A_t \)-Eigenvektoren an. 
\end{enumerate}

\subsection{Ansatz}
\begin{enumerate}
	\item Eine Matrix ist genau dann diagonalisierbar, wenn ihr charakteristisches Polynom in Linearfaktoren zerfällt und für jeden Eigenwert algebraische und geometrische Vielfachheit übereinstimmen.
	\item Eine solche Basis besteht beispielsweise aus den Basisvektoren der Eigenräume. 
\end{enumerate}

\subsection{Lösung}
\begin{enumerate}
	\item Wir bestimmen die Eigenwerte von \( A_t \) über das charakteristische Polynom:
	\begin{equation*}
		\cp_{A_t}(\lambda) = \det(A_t - I_3) = (t-\lambda)((1-t)-\lambda)(1-\lambda) 
	\end{equation*}
	Also ist \( \spec(A_t) = \{ t, 1-t, 1 \} \) und \( A_t \) diagonalisierbar für \( t \in \R \setminus \{ 1, 0, \tfrac{1}{2} \} \).
	\begin{itemize}
		\item \( t = 1 \): Es ist hier \( \mu_a(1) = 2 \). Wir berechnen \( \mu_g(1) \):
		\begin{equation*}
		 	\mu_g(1) = 3 - \rk(A_1 - I_3) = \rk \begin{pmatrix}
		 		1 & 0 & 1 \\
		 		0 & 1 & -1 \\
		 		0 & 0 & 0
		 	\end{pmatrix} = 3-2 = 1\text{.}
		 \end{equation*} 
		 Also ist \( A_1 \) nicht diagonalisierbar.

		 \item \( t = 0 \): Es ist hier \( \mu_a(1) = 2 \). Wie oben:
		 \begin{equation*}
		 	\mu_g(1) = 3 - \rk(A_0 - I_3) = \rk \begin{pmatrix}
		 		1 & 0 & -1 \\
		 		0 & 1 & 1 \\
		 		0 & 0 & 0
		 	\end{pmatrix} = 3-2 = 1
		 \end{equation*}
		 Also ist auch \( A_0 \) nicht diagonalisierbar.

		 \item \( t = \tfrac{1}{2} \): \( \mu_a(\tfrac{1}{2}) = 2 \). Wir berechnen \( \mu_g(\tfrac{1}{2}) = \dim(\kr(A_{1/2} - \tfrac{1}{2}I_3)) \):
		 \begin{equation*}
		 	 \dim(\kr(A_{1/2} - \tfrac{1}{2}I_3)) = \dim \left( \kr\begin{pmatrix}
		 	 	1 & 1 & \tfrac{1}{2} \\
		 	 	0 & 0 & 0 \\
		 	 	0 & 0 & 0
		 	 \end{pmatrix} \right) = \left| \langle \begin{pmatrix}
		 	 	1 \\ 0 \\ -2
		 	 \end{pmatrix}, \begin{pmatrix}
		 	 	0 \\ 1 \\ -2
		 	 \end{pmatrix} \rangle \right| = 2
		 \end{equation*}
		 Wir verfahren analog für den \( E_1 = \langle \begin{pmatrix}
		 	1 & -1 & 1
		 \end{pmatrix}^\top \rangle \) und erhalten so die Basis
		 \begin{equation*}
		 	\left\{ \begin{pmatrix}
		 		1 \\ 0 \\ -2
		 	\end{pmatrix}, \begin{pmatrix}
		 		0 \\ 1 \\ -2
		 	\end{pmatrix}, \begin{pmatrix}
		 		1 \\ -1 \\ 1
		 	\end{pmatrix} \right\}\text{. Insgesamt ist also \( A_{t \in \R \setminus \{ 0, 1 \} } \) diagonalisierbar.}
		 \end{equation*}
	\end{itemize}
\end{enumerate}
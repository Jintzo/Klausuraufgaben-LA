\chapter{Aufgabe 4}

%----------------------------------------------------------------------------------------
%	FRÜHJAHR 2007
%----------------------------------------------------------------------------------------
\section{Frühjahr 2007}

\subsection{Aufgabe}
Im euklidischen Standardvektorraum \( \R^4 \) sei bezüglich der Standardbasis eine Isometrie gegeben durch ihre Abbildungsmatrix
\begin{equation*}
	A = \frac{1}{4}\begin{pmatrix}
		3 & - \sqrt{3} & -1 & \sqrt{3} \\
		\sqrt{3} & 3 & \sqrt{3} & 1 \\
		-1 & -\sqrt{3} & 3 & \sqrt{3} \\
		-\sqrt{3} & 1 & -\sqrt{3} & 3
	\end{pmatrix}\text{.}
\end{equation*}
\begin{enumerate}
	\item Bestimmen Sie die Isometrienormalform \( \widetilde{A} \) von \( A \).
	\item Bestimme \( S \in O(4) \) derart, dass \( \widetilde{A} = S^{-1}AS \).
\end{enumerate}

\subsection{Ansatz}
\begin{enumerate}
	\item Betrachte \( B \coloneqq A + A^\top \), berechne \( \cp_B(X) \) und bestimme daraus die Normalform.
	\item Bestimme Orthonormalbasen der Eigenräume von \( B \). Überprüfe noch die Reihenfolge der letzten beiden Vektoren und gebe anschließend \( S=(b_1,b_2,b_3,b_4) \) an.
\end{enumerate}

\subsection{Lösung}
\begin{enumerate}
	\item Wir berechnen
	\begin{equation*}
		\cp_{A+A^\top}(\lambda) = (\lambda - 2)^2(\lambda-1)^2 \leadsto \spec(A+A^\top) = \{ 1,2 \}\text{.}
	\end{equation*}
	Wir erhalten also
	\begin{equation*}
		\widetilde{A} = \begin{pmatrix}
			1 & & & \\
			& 1 & & \\
			 & & \cos(\omega_1) & -\sin(\omega_1) \\
			 & & \sin(\omega_1) & \cos(\omega_1)
		\end{pmatrix} = \begin{pmatrix}
			1 & & & \\
			& 1 & & \\
			& & \tfrac{1}{2} & -\tfrac{\sqrt{3}}{2} \\
			& & \tfrac{\sqrt{3}}{2} & \tfrac{1}{2}
		\end{pmatrix}\text{.}
	\end{equation*}

	\item Wir berechnen
	\begin{equation*}
		\eig(A+A^\top,2) = \langle \left( \begin{smallmatrix}
			1 \\ 0 \\ -1 \\ 0
		\end{smallmatrix} \right), \left( \begin{smallmatrix}
			0 \\ 1 \\ 0 \\ 1
		\end{smallmatrix} \right) \rangle \eqqcolon \langle c_1,c_2 \rangle \text{,} \quad \eig(A+A^\top,1) = \langle \left( \begin{smallmatrix}
			1 \\ 0 \\ 1 \\ 0
		\end{smallmatrix} \right), \left( \begin{smallmatrix}
			0 \\ 1 \\ 0 \\ -1
		\end{smallmatrix} \right) \rangle \eqqcolon \langle c_3,c_4 \rangle \text{,}
	\end{equation*}
	also sind \( \{ \tfrac{1}{\sqrt{2}}c_1 , \tfrac{1}{\sqrt{2}}c_2 \} \eqqcolon \{ b_1,b_2 \} \) und \( \{ \tfrac{1}{\sqrt{2}}c_3, \tfrac{1}{\sqrt{2}}c_4 \} \eqqcolon \{ b_3,b_4 \} \) Orthonormalbasen von des jeweiligen Eigenraums.
	\\*
	Es ist \( Ab_3 = \tfrac{1}{2}b_3 + \tfrac{\sqrt{3}}{2}b_4 \), also passt die Reihenfolge von \( b_3 \) und \( b_4 \) und wir erhalten
	\begin{equation*}
		S = \frac{1}{\sqrt{2}}\begin{pmatrix}
			1 & 0 & 1 & 0 \\
			0 & 1 & 0 & 1 \\
			-1 & 0 & 1 & 0 \\
			0 & 1 & 0 & -1
		\end{pmatrix}\text{.}
	\end{equation*}
\end{enumerate}

\newpage

%----------------------------------------------------------------------------------------
%	HERBST 2007
%----------------------------------------------------------------------------------------
\section{Herbst 2007}

\subsection{Aufgabe}
\begin{enumerate}
	\item Zeigen Sie, dass für
		\begin{equation*}
			A = \frac{1}{4}\begin{pmatrix}
				\sqrt{3}+2 & \sqrt{3}-2 & -\sqrt{2} \\
				\sqrt{3}-2 & \sqrt{3}+2 & -\sqrt{2} \\
				\sqrt{2} & \sqrt{2} & 2\sqrt{3}
			\end{pmatrix}
		\end{equation*}
		die Abbildung \( \Phi: \R^3 \ni x \mapsto Ax \in \R^3 \) eine Isometrie des euklidischen Standardraums \( \R^3 \) ist.
	\item Bestimmen Sie die Isometrienormalform \( B \) von \( A \).
	\item Bestimmen Sie \( S \in O(3) \) derart, dass \( B = S^{-1}AS \).
\end{enumerate}

\subsection{Ansatz}
\begin{enumerate}
	\item Zeige, dass \( A^\top A = I_3 \).
	\item Bestimme, welche Struktur \( B \) haben muss und beachte, dass die Spur ähnlichkeitsinvariant ist.
	\item Bestimme einen normierten Eigenvektor \( b_1 \) zu \( 1 \in \spec(A) \) und zwei weitere normierte Vektoren aus \( \langle b_1 \rangle^\perp \), sodass diese eine Orthonormalbasis des \( \R^3 \) bilden. Konstruiere aus ihnen \( S \).
\end{enumerate}

\subsection{Lösung}
\begin{enumerate}
	\item \( \Phi \) ist eine Isometrie genau dann wenn \( A^\top A = I_3 \), was der Fall ist.
	\item Es ist \( \spur(A) = 1+\sqrt{3} \), also muss das auch die Spur von \( B \) sein. \( B \) sieht folgendermaßen aus:
	\begin{equation*}
		B = \left(\begin{smallmatrix}
			\pm 1 & & \\
			 & b & -c \\
			 & c & b
		\end{smallmatrix}\right) \text{ mit } 2b \pm 1 = 1+\sqrt{3}\text{, } b^2+c^2=1\text{.}
	\end{equation*}
	Stünde \( -1 \) auf der Diagonalen, so müsste \( b > 1 \) sein, was wegen \( b^2+c^2=1 \) nicht geht, also ist mit \( (\tfrac{\sqrt{3}}{2})+c^2=1 \):
	\begin{equation*}
		B = \begin{pmatrix}
			1 & & \\
			& \tfrac{\sqrt{3}}{2} & -\tfrac{1}{2} \\
			& \tfrac{1}{2} & \tfrac{\sqrt{3}}{2}
		\end{pmatrix}\text{.}
	\end{equation*}

	\item Wir bestimmen einen normierten Eigenvektor zu \( 1 \in \spec(\Phi) \):
	\begin{equation*}
		(A - I_3) = \tfrac{1}{4}\left( \begin{smallmatrix}
			\sqrt{3}-2 & \sqrt{3}-2 & -\sqrt{2} \\
			\sqrt{3}-2 & \sqrt{3}-2 & -\sqrt{2} \\
			\sqrt{2} & \sqrt{2} & 2\sqrt{3}-4
		\end{smallmatrix} \right) \leadsto b_1 \coloneqq \tfrac{\sqrt{2}}{2}\left( \begin{smallmatrix}
			1 \\ -1 \\ 0
		\end{smallmatrix} \right)\text{.}
	\end{equation*}
	Aus dem orthogonalen Komplementärraum wählen wir
	\begin{equation*}
		b_2 \coloneqq \left( \begin{smallmatrix}
			0 \\ 0 \\ 1
		\end{smallmatrix} \right)\text{,} \quad \widetilde{b}_3 \coloneqq Ab_2 = \tfrac{1}{4} \left( \begin{smallmatrix}
			-\sqrt{2} \\ -\sqrt{2} \\ 2\sqrt{3}
		\end{smallmatrix} \right) \leadsto b_3 \coloneqq 2(Ab_2-\tfrac{\sqrt{3}}{2}b_2) = \tfrac{\sqrt{2}}{2}\left( \begin{smallmatrix}
			1 \\ 1 \\ 0
		\end{smallmatrix} \right)\text{}
	\end{equation*}
	Es ist \( Ab_2 = \tfrac{\sqrt{3}}{2}b_2 + \tfrac{1}{2}b_3 \), also passt die Reihenfolge der Basisvektoren und es ist
	\begin{equation*}
		S = (b_1,b_2,b_3) = \begin{pmatrix}
			\tfrac{\sqrt{2}}{2} & 0 & \tfrac{\sqrt{2}}{2} \\
			-\tfrac{\sqrt{2}}{2} & 0 & \tfrac{\sqrt{2}}{2} \\
			0 & 1 & 0
		\end{pmatrix}\text{.}
	\end{equation*}
\end{enumerate}

\newpage

%----------------------------------------------------------------------------------------
%	HERBST 2010
%----------------------------------------------------------------------------------------
\section{Herbst 2010}

\subsection{Aufgabe}
Es sei \( \R^3 \) mit dem Standardskalarprodukt und \( \Phi: \R^3 \to \R^3 \) die orthogonale Spiegelung an der von
\begin{equation*}
	a = \begin{pmatrix}
		1 \\ 1 \\ 1
	\end{pmatrix}\text{,} \quad b = \begin{pmatrix}
		2 \\ -1 \\ 0
	\end{pmatrix}
\end{equation*}
aufgespannten Ebene durch den Ursprung.
\begin{enumerate}
	\item Bestimmen Sie die Abbildungsmatrix von \( \Phi \) bezüglich der Standardbasis des \( \R^3 \).
	\item Bestimmen Sie \( \Phi\left( \left( \begin{smallmatrix}
		-1 \\ 1 \\ 1
	\end{smallmatrix} \right) \right) \).
\end{enumerate}

\subsection{Ansatz}
\begin{enumerate}
	\item Bestimme einen Normalenvektor \( 0 \neq n \in \R^3 \) und verwende diesen in \( \Phi(x) = x - 2\tfrac{\langle x,n \rangle}{\langle n,n \rangle}n \). Bestimme so das Bild eines beliebigen \( x \in \R^3 \), woraus man die Abbildungsmatrix gewinnt.
	\item Es ist \( \Phi(v) = D_{BB}(\Phi)v \) mit der Abbildungsmatrix aus dem ersten Teil.
\end{enumerate}

\subsection{Lösung}
\begin{enumerate}
	\item Sei \( 0 \neq n \in \R^3 \) ein Normalenvektor der Ebene. Dann ist die Spiegelung beschrieben durch
	\begin{equation*}
	 	\Phi(x) = x- 2\tfrac{\langle x,n \rangle}{\langle n,n \rangle}n\text{.}
	 \end{equation*}
	 Zur Bestimmung von \( n \) löst man das LGS \( a^\top n = b^\top n = 0 \). Eine mögliche Lösung ist \( n = \left( \begin{smallmatrix}
	 	1 \\ 2 \\ -3
	 \end{smallmatrix} \right) \). Durch Einsetzen von \( x = \left( \begin{smallmatrix}
	 	x_1 \\ x_2 \\ x_3
	 \end{smallmatrix} \right) \) in \( \Phi(x) \) erhält man
	 \begin{equation*}
	 	\Phi\left( \left( \begin{smallmatrix}
	 		x_1 \\ x_2 \\ x_3
	 	\end{smallmatrix} \right) \right) = \tfrac{1}{7}\left( \begin{smallmatrix}
	 		6x_1-2x_2+3x_3 \\
	 		-2x_1+3x_2+6x_3 \\
	 		3x_1+6x_2-2x_3
	 	\end{smallmatrix} \right)\text{,}
	 \end{equation*}
	 also ist die Abbildungsmatrix von \( \Phi \) (bzgl. Standardbasis \( B \) des \( \R^3 \)):
	 \begin{equation*}
	 	D_{BB}{\Phi} = \frac{1}{7} \begin{pmatrix}
	 		6 & -2 & 3 \\
	 		-2 & 3 & 6 \\
	 		3 & 6 & -2
	 	\end{pmatrix}\text{.}
	 \end{equation*}
	 \item Es ist
	 \begin{equation*}
	 	\Phi\left( \begin{pmatrix}
	 		-1 \\ 1 \\ 1
	 	\end{pmatrix} \right) = D_{BB}(\Phi)\begin{pmatrix}
	 		-1 \\ 1 \\ 1
	 	\end{pmatrix} = \frac{1}{7}\begin{pmatrix}
	 		-5 \\ 11 \\ 1
	 	\end{pmatrix}\text{.}
	 \end{equation*}
\end{enumerate}

\newpage

%----------------------------------------------------------------------------------------
%	FRÜHJAHR 2013
%----------------------------------------------------------------------------------------
\section{Frühjahr 2013}

\subsection{Aufgabe}
Es sei \( \Phi: \R^3 \to \R^3 \) eine Isometrie des euklidischen Standardraums \( \R^3 \) mit \( \det(\Phi) = -1 \).
\begin{enumerate}
	\item Zeigen Sie, dass \( -1 \in \spec(\Phi) \) und dass für jeden Eigenvektor \( v \) zum Eigenwert \( -1 \) gilt:
	\begin{equation*}
	 	\forall x \in \R^3: v \perp (\Phi(x)+x)\text{.}
	 \end{equation*} 
	 \item Für \( \Phi \) gelte zusätzlich
	 \begin{equation*}
	 	\Phi\left( \left( \begin{smallmatrix}
	 		 1 \\ 2 \\ 3
	 	\end{smallmatrix} \right) \right) = \left( \begin{smallmatrix}
	 		\sqrt{2}+2 \\ \sqrt{2}-2 \\ \sqrt{2}
	 	\end{smallmatrix} \right)\text{,} \quad \Phi\left( \left( \begin{smallmatrix}
	 		-1 \\ 3 \\ 2
	 	\end{smallmatrix} \right) \right) = \tfrac{1}{4}\left( \begin{smallmatrix}
	 		\sqrt{2}+12 \\ 6\sqrt{2} -2 \\ \sqrt{2}
	 	\end{smallmatrix} \right)\text{.}
	 \end{equation*}
	 Bestimme die Isometrienormalform von \( \Phi \).
\end{enumerate}

\subsection{Ansatz}
\begin{enumerate}
	\item Weise \( -1 \in \spec(\Phi) \) über die Struktur der Isometrienormalform und der Spur von \( \Phi \), die ähnlichkeitsinvariant ist, nach. Zeige, dass \( \langle v,\Phi(x)+x \rangle = 0 \) gelten muss.
	\item Bestimme den Eigenraum zu \( -1 \) darüber, dass die beiden angegebenen Vektoren offensichtlich im orthogonalen Komplement liegen. Bestimme anschließend \( \angle(y,\Phi(y)) \) und darüber die Isometrienormalform.
\end{enumerate}

\subsection{Lösung}
\begin{enumerate}
	\item Es ist \( -1 \in \spec(\Phi) \), denn wegen \( \det(\Phi) = -1 \) hat auch die Isometrienormalform von \( \Phi \) Determinante \( -1 \) und damit die Gestalt \( \left( \begin{smallmatrix}
	 		-1 & & \\
	 		& \cos(\omega) & -\sin(\omega) \\
	 		& \sin(\omega) & \cos(\omega)
	 	\end{smallmatrix} \right) \).
	 \\* Sei \( v \) Eigenvektor zu \( -1 \), also \( \Phi(v) = -v \). Es ist zu zeigen
	 \begin{align*}
	 	\quad &\langle v,\Phi(x)+x \rangle = 0 \\
	 	\Leftrightarrow \quad &\langle v,x \rangle + \langle v,\Phi(x) \rangle = 0 \\
	 	\overset{\Phi \text{ Iso}}{\Leftrightarrow} \quad &\langle \Phi(v),\Phi(x) \rangle + \langle v,\Phi(x) \rangle = 0 \\
	 	\Leftrightarrow \quad &-\langle v,\Phi(x) \rangle + \langle v,\Phi(x) \rangle = 0\text{.}
	 \end{align*}
	 Also gilt die Behauptung.

	 \item Offensichtlich liegen \( y_1 \coloneqq \left( \begin{smallmatrix}
	 	1 \\ 2 \\ 3
	 \end{smallmatrix} \right) \) und \( y_2 \coloneqq \left( \begin{smallmatrix}
	 	-1 \\ 3 \\ 2
	 \end{smallmatrix} \right) \) nicht im Eigenraum zu \( -1 \) gilt für alle \( v \in \eig(\Phi, -1) \):
	 \begin{equation*}
	 	\langle v,\Phi(y_1)+y_1 \rangle = 0 \quad \text{und} \quad \langle v,\Phi(y_2)+y_2 \rangle = 0\text{.}
	 \end{equation*}
	 Als Lösung des LGS erhalten wir
	 \begin{equation*}
	 	v \in \langle \tfrac{1}{\sqrt{2}}\left( \begin{smallmatrix}
	 		1 \\ 0 \\ -1
	 	\end{smallmatrix} \right) \rangle\text{.}
	 \end{equation*}
	 Wir bestimen nun noch \( \omega = \angle(y,\Phi(y)) \) für \( y \in \langle v \rangle^\perp \):
	 \begin{equation*}
	 	y = \left( \begin{smallmatrix}
	 		1 \\ 2 \\ 3
	 	\end{smallmatrix} \right) - \langle \left( \begin{smallmatrix}
	 		1 \\ 2 \\ 3
	 	\end{smallmatrix} \right), \tfrac{1}{\sqrt{2}}\left( \begin{smallmatrix}
	 		1 \\ 0 \\ -1
	 	\end{smallmatrix} \right) \rangle \tfrac{1}{\sqrt{2}}\left( \begin{smallmatrix}
	 		1 \\ 0 \\ -1
	 	\end{smallmatrix} \right) = \left( \begin{smallmatrix}
	 		2 \\ 2 \\ 2
	 	\end{smallmatrix} \right)
	 \end{equation*}
	 Also ist
	 \begin{equation*}
	 	\cos\angle(y,\Phi(y)) = \tfrac{\langle y,\Phi(y) \rangle}{||y|| \ ||\Phi(y)||} = \tfrac{\sqrt{2}}{2}\text{.}
	 \end{equation*}
	 Also ist die Isometrienormalform von \( A \): 
	 \begin{equation*}
	 	\begin{pmatrix}
	 		-1 & & \\
	 		& \tfrac{\sqrt{2}}{2} & -\tfrac{\sqrt{2}}{2} \\
	 		& \tfrac{\sqrt{2}}{2} & \tfrac{\sqrt{2}}{2}
	 	\end{pmatrix}\text{.}
	 \end{equation*}
\end{enumerate}

\newpage

%----------------------------------------------------------------------------------------
%	HERBST 2013
%----------------------------------------------------------------------------------------
\section{Herbst 2013}

\subsection{Aufgabe}
Seien \( V \) ein endlichdimensionaler euklidischer Vektorraum und \( \Phi, \Psi \in \text{End}(V) \). Für diese gelte \( \Phi \circ \Psi = \Psi \circ \Phi \). Zeigen Sie:
\begin{enumerate}
	\item Jeder \( \Phi \)-Eigenraum ist \( \Psi \)-invariant.
	\item Wenn \( \Phi \) und \( \Psi \) selbstadjungiert sind, dann gibt es eine Orthonormalbasis von \( V \), die aus Eigenvektoren von \( \Phi \) und \( \Psi \) besteht. 
\end{enumerate}

\subsection{Ansatz}
\begin{enumerate}
	\item Zeige, dass für alle \( v \in \eig(\Phi,\lambda) \) auch \( \Psi(v) \in \eig(\Phi,\lambda) \).
	\item Zeige, dass jeder \( \Phi \)-Eigenraum eine Orthonormalbasis aus \( \Psi \)-Eigenvektoren besitzt und die Summe dieser Basen eine Basis von \( V \) ist. 
\end{enumerate}

\subsection{Lösung}
\begin{enumerate}
	\item Sei \( \lambda \in \spec(\Phi) \). Für alle \( v \in \eig(\Phi,\lambda) \) gilt dann
	\begin{equation*}
	 	\Phi(\Psi(v)) = \Psi(\Phi(v)) = \Psi(\lambda v) = \lambda \Psi(v)\text{,}
	 \end{equation*} 
	 also ist auch \( \Psi(v) \in \eig(\Phi,\lambda) \) und damit \( \eig(\Phi,\lambda) \) \( \Psi \)-invariant.

	 \item Da \( \Phi \) selbstadjungiert ist lässt sich \( V \) als orthogonale Summe von \( \Phi \)-Eigenräumen darstellen, welche nach dem ersten Teil \( \Psi \)-invariant sind. \\*
	 	Es ist \( \Psi|_{\eig(\Phi, \lambda) } \) selbstadjungiert, also besitzt jeder Eigenraum eine Orthonormalbasis aus \( \Psi \)-Eigenvektoren, welche auch Eigenvektoren von \( \Phi \) sind. Die Summe dieser Orthonormalbasen ist eine Orthonormalbasis von \( V \).
\end{enumerate}

\newpage

%----------------------------------------------------------------------------------------
%	FRÜHJAHR 2014
%----------------------------------------------------------------------------------------
\section{Frühjahr 2014}

\subsection{Aufgabe}
Seien \( 2 \leq n \in \N \), \( V \) ein \( n \)-dimensionaler euklidischer Vektorraum mit Skalarprodukt \( \langle \cdot, \cdot \rangle \) und \( w_1, w_2 \in V \setminus \{ 0 \} \) mit \( \langle w_1,w_2 \rangle \neq 0 \). Weiter sei der Endomorphismus \( \Phi \) gegeben:
\begin{equation*}
	\Phi: V \ni v \mapsto \langle v,w_1 \rangle w_2 \in V
\end{equation*}
\begin{enumerate}
	\item Geben Sie \( \bild(\Phi) \) und \( \rk(\Phi) \) an.
	\item Bestimmen Sie alle \( \lambda \in \spec(\Phi) \) sowie die zugehörigen Eigenräume.
	\item Bestimmen Sie die zu \( \Phi \) adjungierte Abbildung \( \Phi^\ast \).
	\item Zeigen Sie: \( \Phi \) selbstadjungiert \( \Leftrightarrow \) \( w_1 \) und \( w_2 \) sind linear abhängig.
\end{enumerate}

\subsection{Ansatz}
\begin{enumerate}
	\item Zeige, dass \( \bild(\Phi) \) nicht leer ist und von \( w_2 \) erzeugt wird. Beachte, dass \( \rk(\Phi) = \dim(\bild(\Phi)) \).
	\item Berechne \( \kr(\Phi) \), woraus man \( 0 \in \spec(\Phi) \) erhält. Bestimme die übrigen Eigenwerte aus der Abbildungsvorschrift.
	\item Forme \( \langle v_1, \Phi(v_2) \rangle \) um und konstruiere daraus eine Abbildung \( \Phi^\ast \).
	\item Zeige die beiden Richtungen einzeln: Verwende von links nach rechts \( \Phi \overset{\text{!}}{=} \Phi^\ast \) aus Teil 3 und für die Rückrichtung \( w_1 = \lambda w_2 \) in \( \langle v,w_1 \rangle w_2 \).
\end{enumerate}

\subsection{Lösung}
\begin{enumerate}
	\item Wir können ablesen, dass \( \bild(\Phi) \) in \( \langle w_2 \rangle \) enthalten ist und \( 0 \neq \langle w_2,w_1 \rangle \in \bild(\Phi) \), also \( \bild(\Phi) = \langle w_2 \rangle \) und \( \rk(\Phi) = \dim(\bild(\Phi)) = 1 \).
	\item Es ist \( \dim(\kr(\Phi)) = \dim(V) - \rk(\Phi) = n-1 \), also ist \( 0 \in \spec(\Phi) \) \( (n-1) \)-facher Eigenwert. Außerdem ist \( \Phi(w_2) = \langle w_2,w_1 \rangle w_2 \), also ist \( \langle w_1,w_2 \rangle \) einfacher Eigenwert von \( \Phi \). \\
	Wir erhalten \( \eig(\Phi,0) = \langle w_1 \rangle^\perp \), \( \eig(\Phi,\langle w_1,w_2 \rangle) = \langle w_2 \rangle \).
	\item Für \( v_1,v_2 \in V \) gilt
	\begin{equation*}
		\langle v_1, \Phi(v_2) \rangle = \langle v_1,\langle v_2,w_1 \rangle w_2 \rangle = \langle v_2,w_1 \rangle \langle v_1,w_2 \rangle = \langle v_2,w_1\langle v_1,w_2 \rangle \rangle = \langle \langle v_1,w_2 \rangle w_1,v_2 \rangle\text{.}
	\end{equation*}
	Also ist \( \Phi^\ast: V \ni v \mapsto \langle v,w_2 \rangle w_1 \in V \) die zu \( \Phi \) adjungierte Abbildung.

	\item \( \Rightarrow \): \( \Phi \) ist selbstadjungiert genau dann, wenn \( \langle v,w_1 \rangle w_2 = \langle v,w_2 \rangle w_1 \), also muss dann \( w_1 = \tfrac{\langle w_1,w_1 \rangle}{\langle w_1,w_2 \rangle}w_2 \) sein, also sind \( w_1 \) und \( w_2 \) linear abhängig. \\*
	\( \Leftarrow \): Ist \( w_1 = \lambda w_2 \), dann ist \( \langle v,w_1 \rangle w_2 = \lambda \langle v,w_2 \rangle w_2 = \langle v,w_2 \rangle w_1 \) und somit ist \( \Phi \) selbstadjungiert.
\end{enumerate}

\newpage

%----------------------------------------------------------------------------------------
%	HERBST 2014
%----------------------------------------------------------------------------------------
\section{Herbst 2014}

\subsection{Aufgabe}
Es sei \( (V,\langle \cdot, \cdot \rangle) \) ein euklidischer Vektorraum der Dimension \( n \geq 1 \), und \( f \in \text{End}(V) \) mit der Eigenschaft
\begin{equation*}
	\forall x \in V: \langle f(x),x \rangle = 0\text{.}
\end{equation*}
Zeigen Sie:
\begin{enumerate}
	\item \( \forall x,y \in V : \langle f(x),y \rangle = -\langle x,f(y) \rangle \).
	\item Ist \( U \leq V \) \( f \)-invariant, so ist auch \( U^\perp \) \( f \)-invariant.
	\item Besitzt \( f \) einen Eigenwert \( \lambda \in \R \), so ist \( f \) nicht invertierbar.
	\item Der Endomorphismus \( f^2: V \ni x \mapsto f(f(x)) \in V \) ist selbstadjungiert.
	\item Es gibt eine Orthonormalbasis \( B \) von \( V \) und reelle Zahlen \( \mu_1, \dots, \mu_n \leq 0 \), sodass die Abbildungsmatrix von \( f^2 \) bezüglich \( B \) gegeben ist durch
	\begin{equation*}
	 	D_{BB}(f^2) = \begin{pmatrix}
	 		\mu_1 & & \\
	 		& \ddots & \\
	 		& & \mu_n
	 	\end{pmatrix}\text{,}
	 \end{equation*} 
\end{enumerate}

\subsection{Ansatz}
\begin{enumerate}
	\item Zeige durch Umformen, dass \( \langle f(y),x \rangle + \langle f(x),y \rangle = 0 \).
	\item Zeige, dass für \( y \in U^\perp, u \in U \) gilt: \( \langle u,f(y) \rangle = 0 \).
	\item Zeige, dass \( f \) nicht injektiv und somit nicht invertierbar ist.
	\item Zeige \( \langle f^2(x),y \rangle = \langle x,f^2(y) \rangle \) durch Umformen.
	\item Offensichtlich existiert eine \( V \)-ONB aus \( f^2 \)-Eigenvektoren, da \( f^2 \) selbstadjungiert ist. Zeige, dass die zugehörigen Eigenwerte \( \leq 0 \) sind.
\end{enumerate}

\subsection{Lösung}
\begin{enumerate}
	\item Sei \( x,y \in V \). Dann ist
	\begin{align*}
		\quad &\langle f(y),x \rangle = \langle f(x),y \rangle \\
		\Leftrightarrow \quad &\langle f(y),x \rangle + \langle f(x),y \rangle = 0 \\
		\Leftrightarrow \quad &\langle f(y),x \rangle + \langle f(x),y \rangle + \langle f(x),x \rangle + \langle f(y),y \rangle = 0 \\
		\Leftrightarrow \quad &\langle f(x+y),x+y \rangle = 0\text{,}
	\end{align*}
	also gilt die Behauptung.
	\item Sei \( y \in U^\perp \), dann gilt für alle \( u \in U \), dass \( 0 = \langle f(u),y \rangle = -\langle u,f(y) \rangle \), also \( f(y) \in U^\perp \).
	\item Sei \( 0 \neq x \in \eig(f,\lambda) \). Dann
	\begin{equation*}
	 	0 = \langle f(x),x \rangle = \langle \lambda x,x \rangle = \lambda \langle x,x \rangle\text{,}
	 \end{equation*} 
	 also muss \( \lambda = 0 \) sein und damit \( \kr(f) = \eig(f,0) \neq \{ 0 \} \), also ist \( f \) nicht injektiv und damit insbesondere nicht invertierbar.
	\item Seien \( x,y \in V \). Dann
	\begin{equation*}
		\langle f^2(x),y \rangle = \langle f(f(x)),y \rangle = - \langle f(x),f(y) \rangle = (-1)^2\langle x,f(f(y)) \rangle = \langle x,f^2(y) \rangle
	\end{equation*}
	und somit ist \( f \) selbstadjungiert.
	\item \( f^2 \) ist selbadjungiert, also existiert eine ONB \( B = (b_1, \dots, b_n) \) aus \( f^2 \)-Eigenvektoren. Für die zugehörigen Eigenwerte gilt
	\begin{equation*}
		\mu_i = \langle \mu_ib_i,b_i \rangle = \langle f^2(b_i),b_i \rangle - \langle f(b_i),f(b_i) \rangle \leq 0\text{.}
	\end{equation*}
\end{enumerate}

\newpage

%----------------------------------------------------------------------------------------
%	FRÜHJAHR 2015
%----------------------------------------------------------------------------------------
\section{Frühjahr 2015}

\subsection{Aufgabe}
Es sei \( (V,\langle \cdot,\cdot \rangle) \) ein endlichdimensionaler euklidischer Vektorraum. Weiter seien \( \Phi,\Psi \in \text{End}(V) \) selbstadjungiert.
\begin{enumerate}
	\item Geben Sie die Definition von ``selbstadjungiert'' an.
	\item Zeigen Sie, dass sowohl \( \Phi + \Psi \) als auch \( \Phi - \Psi \) selbstadjungiert sind.
	\item Zeigen Sie: \( \Phi \circ \Psi \) ist selbstadjungiert \( \Leftrightarrow \Phi \circ \Psi = \Psi \circ \Phi \).
\end{enumerate}

\subsection{Ansatz}
\begin{enumerate}
	\item Gib die Definition an. 
	\item Zeige, durch Umformungen, dass \( \langle (\Phi + a\Psi)(v),w \rangle = \langle v,(\Phi + a\Psi)w \rangle \), \( a \in \R \), gilt.
	\item Zeige beide Richtungen einzeln jeweils durch Umformen.
\end{enumerate}

\subsection{Lösung}
\begin{enumerate}
	\item Ein Endomorphismus \( \Phi \) eines endlichdimensionalen euklidischen Vektorraums \( V, \langle \cdot,\cdot \rangle \) heißt selbstadjungiert, falls für alle \( v,w \in V \) gilt: \( \langle \Phi(v),w \rangle = \langle v,\Phi(w) \rangle \).
	\item Wir zeigen, dass \( \Phi + a\Psi \) mit \( a \in \R \) selbstadjungiert ist:
	\begin{align*}
		\langle (\Phi + a\Psi)(v),w \rangle &= \langle \Phi(v),w \rangle + a\langle \Psi(v),w \rangle \\
		&= \langle v,\Phi(w) \rangle + a\langle v,\Psi(w) \rangle \\
		&= \langle v,(\Psi + a\Psi)(w) \rangle\text{.}
	\end{align*}
	\item \( \Rightarrow \): Es ist
	\begin{align*}
		\quad &\langle \Phi(\Psi(v)),w \rangle - \langle \Phi(\Psi(v)),w \rangle = 0 \\
		\Leftrightarrow \quad &\langle \Phi(\Psi(v)),w \rangle - \langle \Phi(v),\Psi(w) \rangle = 0 \\
		\Leftrightarrow \quad &\langle \Phi(\Psi(v)),w \rangle - \langle \Psi(\Phi(v)),w \rangle = 0 \\
		\Leftrightarrow \quad &\langle \Phi(\Psi(v))-\Psi(\Phi(v)),w \rangle = 0
	\end{align*}
	Also muss \( \Phi(\Psi(v)) =\Psi(\Phi(v)) \) sein.
	\\*
	\( \Leftarrow \): Es ist
	\begin{align*}
		\quad &\langle \Phi(\Psi(v)),w \rangle = \langle \Psi(\Phi(v),w) \rangle \\
		\Leftrightarrow \quad &\langle \Phi(\Psi(v)),w \rangle = \langle \Phi(v),\Psi(w) \rangle \\
		\Leftrightarrow \quad &\langle \Phi(\Psi(v)),w \rangle = \langle v,\Phi(\Psi(w)) \rangle
	\end{align*}
	Alsi ist \( \Phi \circ \Psi \) selbstadjungiert.
\end{enumerate}
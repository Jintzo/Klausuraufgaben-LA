\chapter{Aufgabe 1}

%----------------------------------------------------------------------------------------
%	FRÜHJAHR 2007
%----------------------------------------------------------------------------------------
\section{Frühjahr 2007}

\subsection{Aufgabe}
Gegeben sei die reelle Matrix
\begin{equation*}
	A = \begin{pmatrix}
		-1 & 0 & -1 & 0 \\
		1 & -1 & 1 & 2 \\
		1 & 0 & 1 & 1 \\
		0 & 0 & 1 & -1
	\end{pmatrix}\text{.}
\end{equation*}
\begin{enumerate}
	\item Berechnen Sie die JNF von \( A \).
	\item Zeigen Sie, dass es keine Matrix \( B \) mit folgenden Eigenschaften gibt:
	\begin{enumerate}
	 	\item \( A \) und \( B \) sind ähnlich.
	 	\item Es ist \( B^2 = \left( \begin{smallmatrix}
	 		1 & 0 & 0 & 0 \\
	 		1 & 1 & 0 & 0 \\
	 		0 & 0 & 1 & 0 \\
	 		0 & 0 & 0 & 1
	 	\end{smallmatrix} \right) \) 
	 \end{enumerate} 
\end{enumerate}

\subsection{Ansatz}
\begin{enumerate}
	\item Berechne \( \cp_A(\lambda) \). Berechne  \( \mu_a(\lambda) \) und \( \mu_g(\lambda) \) für alle \( \lambda \in \spec(A) \) und konstruiere so \( A' \coloneqq \jnf(A) \).
	\item Für eine solche Matrix \( B \) muss \( \jnf(A'^2) \sim \jnf(B^2) \) gelten (dies ist zuerst zu zeigen). Zeige, dass die Anzahl der Jordankästchen zu einem Eigenwert unterschiedlich ist.
\end{enumerate}

\subsection{Lösung}
\begin{enumerate}
	\item Es ist
	\begin{equation*}
	 	\cp_A(\lambda) = (-1-\lambda)^3(1-\lambda)\text{.}
	 \end{equation*}
	 Wir bestimmen \( \mu_g(-1) \):
	 \begin{equation*}
	 	\mu_g(-1) = \dim(\eig(A, -1)) = \dim(\kr(A + I_4)) = \dim\left( \kr \left( \begin{smallmatrix}
	 		0 & 0 & -1 & 0 \\
	 		1 & 0 & 1 & 2 \\
	 		1 & 0 & 2 & 1 \\
	 		0 & 0 & 1 & 0
	 	\end{smallmatrix} \right) \right) = 1
	 \end{equation*}
	 Also gibt es zu \( \lambda = -1 \) genau ein Jordankästchen und wir erhalten die JNF:
	 \begin{equation*}
	 	A' \coloneqq \left( \begin{smallmatrix}
	 		-1 & 0 & 0 & 0 \\
	 		1 & -1 & 0 & 0 \\
	 		0 & 1 & -1 & 0 \\
	 		0 & 0 & 0 & 1
	 	\end{smallmatrix} \right)\text{.}
	 \end{equation*}

	 \item Existiert ein solches \( B \), so gilt
	 \begin{equation*} 
	 	A \sim B \Rightarrow A'^2 \sim B^2 \Rightarrow \jnf(A'^2) \sim \jnf(B^2) = \left( \begin{smallmatrix}
	 		1 & 0 & 0 & 0 \\
	 		1 & 1 & 0 & 0 \\
	 		0 & 0 & 1 & 0 \\
	 		0 & 0 & 0 & 1
	 	\end{smallmatrix} \right)\text{.}
	 \end{equation*}
	 Nun ist aber \( \mu_g(1) = 2 \) in \( A'^2 = \left( \begin{smallmatrix}
	 	1 & 0 & 0 & 0 \\
	 	-2 & 1 & 0 & 0 \\
	 	1 & -2 & 1 & 0 \\
	 	0 & 0 & 0 & 1 \\
	 \end{smallmatrix} \right) \), also hat \( \jnf(A'^2) \) genau zwei Jordankästchen zum Eigenwert \( 1 \), \( \jnf(B^2) \) aber drei, also kann es eine solche Matrix \( B \) nicht geben.
\end{enumerate}

\newpage

%----------------------------------------------------------------------------------------
%	HERBST 2007
%----------------------------------------------------------------------------------------
\section{Herbst 2007}

\subsection{Aufgabe}
Es seien \( V \) ein \( n \)-dimensionaler komplexer Vektorraum und \( \Phi \in \text{End}(\Phi) \).
\begin{enumerate}
	\item Zeigen Sie, dass \( V \) die direkte Summe der Untervektorräume \( \kr(\Phi^n) \) und \( \bild(\Phi^n) \) ist. 
	\item Geben Sie ein Beispiel dafür an, dass \( V \) i.A. nicht die direkte Summe von \( \kr(\Phi) \) und \( \bild(\Phi) \) ist.
\end{enumerate}

\subsection{Ansatz}
\begin{enumerate}
	\item Betrachte den Dimensionssatz und \( \kr(\Phi^n) \cap \im(\Phi^n) \).
	\item Konstruiere ein \( \Phi \in \text{End}(\C^2) \), für die \( \kr(\Phi) = \im(\Phi) \) gilt. 
\end{enumerate}

\subsection{Lösung}
\begin{enumerate}
	\item Wir müssen zeigen, dass \( \kr(\Phi^n) \cap \im(\Phi^n) = \{ 0 \} \), denn dann folgt aus der Dimensionsformel, dass \( \im(\Phi^n) + \kr(\Phi^n) = \im(\Phi^n) \oplus \kr(\Phi^n) = V \). \\
		Es ist \( \kr(\Phi^n) = \hau(\Phi, 0) = \{ v \in V \mid \exists k \in \N: \Phi^k(v) = 0 \} \). Es liege \( v \in \kr(\Phi^n) \cap \im(\Phi^n) \). Also gilt
		\begin{equation*}
			\exists w \in V: \Phi^n(w) = v \wedge \Phi^n(v) = 0 \quad \leadsto \quad \Phi^n(\Phi^n(w)) = \Phi^{2n}(w) = 0
		\end{equation*}
		Daraus folgt
		\begin{equation*}
			\Phi^n(w) = 0 = v \in \kr(\Phi^n)
		\end{equation*}
		und damit ist die Behauptung gezeigt. 

	\item Betrachte die Abbildung
		\begin{equation*}
			\text{End}(\C^2) \ni \Phi\left( \begin{pmatrix}
				x \\ y
			\end{pmatrix} \right) \mapsto \begin{pmatrix}
				y \\ 0
			\end{pmatrix}\text{.}
		\end{equation*}
		Es gilt
		\begin{equation*}
			\kr(\Phi) = \C e_1 = \im(\Phi)\text{,}
		\end{equation*}
		also kann die Summe nicht direkt sein.
\end{enumerate}

\newpage

%----------------------------------------------------------------------------------------
%	HERBST 2010
%----------------------------------------------------------------------------------------
\section{Herbst 2010}

\subsection{Aufgabe}
Es sei \( A \in \C^{n \times n} \) gegeben und \( M \in \C^{(2n) \times (2n)} \) die Blockmatrix
\begin{equation*}
	M \coloneqq \begin{pmatrix}
		0 & 0 \\
		A & 0 
	\end{pmatrix}\text{.}
\end{equation*}
\begin{enumerate}
	\item Bestimme \( \spec(M) \).
	\item Wie lang sind die längsten Jordankästchen in \( \jnf(M) \)?
	\item Bestimme \( \jnf(M) \) in Abhängigkeit von \( \rk(A) \).
	\item Bestimme für \( A = \left( \begin{smallmatrix}
		1 & 2 \\
		0 & 1
	\end{smallmatrix} \right) \) eine Basiswechselmatrix \( S \in \gl_4(\C) \) mit \( S^{-1}MS = \jnf(M) \).
\end{enumerate}

\subsection{Ansatz}
\begin{enumerate}
	\item Betrachte die Diagonale der Matrix.
	\item Die länge \( p \) des längsten Jordankästchens zum Eigenwert \( \lambda \) ist \\* \( \min\{ e \in \N \mid \kr((M  - \lambda I_{2n})^e) = \kr((M - \lambda I_{2n})^{e+1}) \} \).
	\item Verwende \( \mu_g(\lambda) = 2n - \rk(M-\lambda I_{2n}) \).
	\item Bestimme für jeden Eigenwert eine Basis der Kerne von \( (M - \lambda I_{2n})^k \) (\( 1 \leq k \leq p \), siehe 2.) und füge die Vektoren \( (M - \lambda I_{2n})^kv \) (\( 1 \leq k \leq p \), \( v \in (\kr((M - \lambda I_{2n})^p) \setminus \kr((M - \lambda I_{2n})^{p-1})) \)) der Matrix hinzu.
\end{enumerate}

\subsection{Lösung}
\begin{enumerate}
	\item \( M \) ist eine strikte untere Dreiecksmatrix, also ist \( \spec(M) = \{ 0 \} \).

	\item Da \( \lambda = 0 \) der einzige Eigenwert ist, ist die Länge des größten Jordankästchens das kleinste \( k \in \N \) mit \( M^k = 0 \). Es ist
		\begin{equation*}
		 	k = \begin{cases}
		 		2 &\text{falls } A \neq 0 \\
		 		1 &\text{sonst}
		 	\end{cases}
		 \end{equation*} 
	\item Die Anzahl der Jordankästchen zu \( 0 \) ist \( \mu_g(0) \). Es ist
		\begin{equation*}
			\mu_g(0) = \dim(\eig(M, 0)) = 2n - \rk(M - 0*I_{2n}) = 2n - \rk(A)\text{.}
		\end{equation*}
		Die Summe der Kästchenlängen ist offensichtlich \( 2n \).
		\begin{itemize}
			\item Anzahl \( 2 \)er-Kästchen: \( = \rk(A) \) (damit \( 2n-\rk(M) = 2n-\rk(A) \) Kästchen eine Gesamtlänge von \( 2n \) haben)
			\item Anzahl \( 1 \)er-Kästchen: \( = \underbrace{2n-\rk(A)}_{\text{= \# Kästchen ges.}} - \underbrace{\rk(A)}_{\text{\#2er-Kästchen}} = 2n - 2\rk(A) \)
		\end{itemize}
		Also ist \( \jnf(M) \) bis auf die Reihenfolge der Kästchen eindeutig bestimmt.

	\item Wir bestimmen Basen für folgende Kerne (\( \lambda = 0 \) ist einziger Eigenwert und \( M^2 = 0 \), deswegen nur die zwei):
		\begin{equation*}
			\kr(M - 0*I_4)=\kr(M), \quad \kr((M-0*I_4)^2) = \kr(M^2)\text{.}
		\end{equation*}
		Wir erhalten
		\begin{equation*}
			\kr(M) = \langle \left( \begin{smallmatrix}
				0 \\ 0 \\ 1 \\ 0
			\end{smallmatrix} \right), \left( \begin{smallmatrix}
				0 \\ 0 \\ 0 \\ 1
			\end{smallmatrix} \right) \rangle, \quad \kr(M^2) = \langle \left( \begin{smallmatrix}
				1 \\ 0 \\ 0 \\ 0
			\end{smallmatrix} \right), \left( \begin{smallmatrix}
				0 \\ 1 \\ 0 \\ 0
			\end{smallmatrix} \right), \left( \begin{smallmatrix}
				0 \\ 0 \\ 1 \\ 0
			\end{smallmatrix} \right), \left( \begin{smallmatrix}
				0 \\ 0 \\ 0 \\ 1
			\end{smallmatrix} \right) \rangle\text{.}
		\end{equation*}
		Wir wählen die Basisvektoren von \( \kr(M^2) \setminus \kr(M) \), also \( v_1 = \left( \begin{smallmatrix}
			1 \\ 0 \\ 0 \\ 0
		\end{smallmatrix} \right) \) und \( v_2 = \left( \begin{smallmatrix}
			0 \\ 1 \\ 0 \\ 0
		\end{smallmatrix} \right) \).
		\begin{equation*}
			S = \left( v_1, (A - 0*I_4)v_1, v_2, (A-0*I_4)v_2 \right) = \left( \left( \begin{smallmatrix}
			1 \\ 0 \\ 0 \\ 0
		\end{smallmatrix} \right), \left( \begin{smallmatrix}
			0 \\ 0 \\ 1 \\ 0
		\end{smallmatrix} \right), \left( \begin{smallmatrix}
			0 \\ 1 \\ 0 \\ 0
		\end{smallmatrix} \right), \left( \begin{smallmatrix}
			0 \\ 0 \\ 2 \\ 1
		\end{smallmatrix} \right) \right)\text{.}
		\end{equation*}
\end{enumerate}

\newpage

%----------------------------------------------------------------------------------------
%	FRÜHJAHR 2013
%----------------------------------------------------------------------------------------
\section{Frühjahr 2013}

\subsection{Aufgabe}
Gegeben sei
\begin{equation*}
	A \coloneqq \begin{pmatrix}
		1 & 0 & -1 & -1 \\
		-1 & 0 & 0 & 0 \\
		0 & 0 & 2 & 1 \\
		1 & 1 & 0 & 1
	\end{pmatrix} \in \R^{4 \times 4}\text{.}
\end{equation*}
\begin{enumerate}
	\item Bestimme \( \widetilde{A} \coloneqq \jnf(A) \).
	\item Geben Sie eine invertierbare Matrix \( S \in \R^{4 \times 4} \) an, sodass \( \widetilde{A} = S^{-1}AS \). 
\end{enumerate}

\subsection{Ansatz}
\begin{enumerate}
	\item Berechne \( \cp_A(\lambda) \), dann \( \mu_a(\lambda) \) und \( \mu_g(\lambda) \) für alle \( \lambda \in \spec(A) \) und konstruiere so \( \jnf(A) \).
	\item Füge für das \( k \) große Kästchen zu \( \lambda \) Basisvektoren \( b_1,\dots \) aus \( \kr((A- \lambda I_4)^k) \) \\* (bzw. \( (A-\lambda I_4)^lb_i \), \( 0 \leq l \leq (k-1) \)) zur Basis hinzu.
\end{enumerate}

\subsection{Lösung}
\begin{enumerate}
	\item  Wir berechnen das charakteristische Polynom:
		\begin{equation*}
			\cp_A(X) = \det \left( \begin{smallmatrix}
				1-X & 0 & -1 & -1 \\
				-1 & -X & 0 & 0 \\
				0 & 0 & 2-X & 1 \\
				1 & 1 & 0 & 1-X
			\end{smallmatrix} \right) = (X-1)^4\text{.}
		\end{equation*}
		Es ist also \( \spec(A)= \{ 1 \} \). Die Anzahl an Jordankästchen für den Eigenwert \( \lambda = 1 \) ist
		\begin{equation*}
			\mu_g(1) = \dim(\eig(A,1)) = 4 - \rk(A-I_4) = 4-2 = 2\text{.}
		\end{equation*}
		Um das größte Jordankästchen zu bestimmen berechnen wir
		\begin{align*}
			\dim\kr(A-I_4) =\dim\kr\left( \begin{smallmatrix}
				0 & 0 & -1 & -1 \\
				-1 & -1 & 0 & 0 \\
				0 & 0 & 1 & 1 \\
				1 & 1 & 0 & 0
			\end{smallmatrix} \right) = \dim \langle \left( \begin{smallmatrix}
				1 \\ -1 \\ 0 \\ 0
			\end{smallmatrix} \right), \left( \begin{smallmatrix}
				0 \\ 0 \\ 1 \\ -1
			\end{smallmatrix} \right) \rangle &= 2\text{,} \\
			\dim\kr((A-I_4)^2) = \dim\kr\left( \begin{smallmatrix}
				-1 & -1 & -1 & -1 \\
				1 & 1 & 1 & 1 \\
				1 & 1 & 1 & 1 \\
				-1 & -1 & -1 & -1
			\end{smallmatrix} \right) = \dim \langle \left( \begin{smallmatrix}
				1 \\ 0 \\ 0 \\ -1
			\end{smallmatrix} \right), \left( \begin{smallmatrix}
				1 \\ 0 \\ -1 \\ 0
			\end{smallmatrix} \right), \left( \begin{smallmatrix}
				1 \\ -1 \\ 0 \\ 0
			\end{smallmatrix} \right) \rangle &= 3\text{,} \\
			\dim\kr((A-I_4)^3) = \dim\kr\left( \begin{smallmatrix}
				0 & 0 & 0 & 0 \\
				0 & 0 & 0 & 0 \\
				0 & 0 & 0 & 0 \\
				0 & 0 & 0 & 0
			\end{smallmatrix} \right) = \dim(\R^4) &= 4\text{.}
		\end{align*}
		Das größte Jordankästchen hat also die Größe \( 3 \) und somit
		\begin{equation*}
			\widetilde{A} = \left( \begin{array}{ccc|c}
				1 & 0 & 0 & 0 \\
				1 & 1 & 0 & 0 \\
				0 & 1 & 1 & 0 \\
				\hline
				0 & 0 & 0 & 1

			\end{array} \right)\text{.}
		\end{equation*}

	\item Für das \( 3 \)er-Kästchen nehmen wir einen Vektor aus \( \kr((A-I_4)^3) \setminus \kr((A-I_4)^2) \), also z.B. \( e_1 \), und fügen \( e_1 \), \( (A-I_4)e_1 \) und \( (A-I_4)^2e_1 \) der Basis hinzu.
		\\*
		Für das \( 1 \)er-Kästchen nehmen wir einen Vektor aus \( \kr(A-I_4) \), der noch nicht in der Basis liegt, also z.B. \( \left( \begin{smallmatrix}
			1 \\ -1 \\ 0 \\ 0
		\end{smallmatrix} \right) \).
		Wir erhalten 
		\begin{equation*}
			S = \begin{pmatrix}
				1 & 0 & -1 & 1 \\
				0 & -1 & 1 & -1 \\
				0 & 0 & 1 & 0 \\
				0 & 1 & -1 & 0
			\end{pmatrix}\text{.}
		\end{equation*}
\end{enumerate}

\newpage

%----------------------------------------------------------------------------------------
%	HERBST 2013
%----------------------------------------------------------------------------------------
\section{Herbst 2013}

\subsection{Aufgabe}
Gegeben sei
\begin{equation*}
	A \coloneqq \begin{pmatrix}
		1 & 0 & 0 & 1 \\
		3 & 1 & -3 & 1 \\
		3 & 0 & -2 & 1 \\
		0 & -1 & 1 & 1 
	\end{pmatrix} \in \R^{4 \times 4}\text{.}
\end{equation*}
\begin{enumerate}
	\item Bestimmem Sie \( \widetilde{A} \coloneqq \jnf(A) \).
	\item Geben Sie \( S \in \gl_4(\R) \) an, sodass \( \widetilde{A} = S^{-1}AS \).
\end{enumerate}

\subsection{Ansatz}
\begin{enumerate}
	\item Berechne \( \cp_A(\lambda) \), dann \( \mu_a(\lambda) \) und \( \mu_g(\lambda) \) für alle \( \lambda \in \spec(A) \) und konstruiere so \( \jnf(A) \).
	\item Füge für das \( k \) große Kästchen zu \( \lambda \) Basisvektoren \( b_1,\dots \) aus \( \kr((A-\lambda I_4)^k) \) \\* (bzw. \( (A-\lambda I_4)^lb_i \), \( 0 \leq l \leq (k-1) \)) zur Basis hinzu.
\end{enumerate}

\subsection{Lösung}
\begin{enumerate}
	\item Wir bestimmen
		\begin{equation*}
		 	\cp_A(\lambda) = (\lambda + 2)(\lambda - 1)^3 \leadsto \spec(A) = \{ -2, 1 \}, \ \mu_a(-2) = \mu_g(-2)=1, \ \mu_a(1)=3\text{,}
		 \end{equation*}
		 anschließend noch
		 \begin{equation*}
		 	\mu_g(1) = \dim\kr(A-I_4) = \dim\kr\left( \begin{smallmatrix}
		 		1 & 0 & -1 & 0 \\
		 		0 & 1 & -1 & 0 \\
		 		0 & 0 & 0 & 0 \\
		 		0 & 0 & 0 & 1
		 	\end{smallmatrix} \right) = \dim \langle \left( \begin{smallmatrix}
		 		1 \\ 1 \\ 1 \\ 0
		 	\end{smallmatrix} \right) \rangle = 1\text{.}
		 \end{equation*}
		 Es gibt zu jedem Eigenwert also genau ein Jordankästchen und wir erhalten
		 \begin{equation*}
		 	\widetilde{A} = \left( \begin{array}{ccc|c}
		 		1 & 0 & 0 & 0 \\
		 		1 & 1 & 0 & 0 \\
		 		0 & 1 & 1 & 0 \\
		 		\hline
		 		0 & 0 & 0 & -2
		 	\end{array} \right)
		 \end{equation*}
	\item Das größte Jordankästchen zu \( \lambda = 1 \) ist \( 3 \) groß, also berechnen Basen folgender Kerne:
		\begin{align*}
			\kr((A-I_4)^2) &= \kr\left( \begin{smallmatrix}
				0 & -1 & 1 & 0 \\
				-9 & -1 & 10 & 0 \\
				-9 & -1 & 10 & 0 \\
				0 & 0 & 0 & 0
			\end{smallmatrix} \right) = \langle \left( \begin{smallmatrix}
				1 \\ 1 \\ 1 \\ 0
			\end{smallmatrix} \right), \left( \begin{smallmatrix}
				0 \\ 0 \\ 0 \\ 1
			\end{smallmatrix} \right) \rangle \\
			\kr((A-I_4)^3) &= \kr\left( \begin{smallmatrix}
				0 & 0 & 0 & 0 \\
				27 & 0 & -27 & 0 \\
				27 & 0 & -27 & 0 \\
				0 & 0 & 0 & 0
			\end{smallmatrix} \right) = \langle \left( \begin{smallmatrix}
				1 \\ 0 \\ 1 \\ 0
			\end{smallmatrix} \right), \left( \begin{smallmatrix}
				0 \\ 1 \\ 0 \\ 0
			\end{smallmatrix} \right), \left( \begin{smallmatrix}
				0 \\ 0 \\ 0 \\ 1
			\end{smallmatrix} \right) \rangle
		\end{align*}
		Wir fügen für das \( 3 \)er-Jordankästchen \( b_1 \coloneqq \left( \begin{smallmatrix}
			0 \\ 1 \\ 0 \\ 0
		\end{smallmatrix} \right) \), \( (A-I_4)b_1 \) und \( (A-I_4)^2b_1 \) zur Basis hinzu.
		\\*
		Für \( -2 \) wählen wir einen Vektor aus \( \kr(A+2I_4) \):
		\begin{equation*}
			\kr(A+2I_2) = \kr\left( \begin{smallmatrix}
				3 & 0 & 0 & 1 \\
				3 & 3 & -3 & 1 \\
				3 & 0 & 0 & 1 \\
				0 & -1 & 1 & 3
			\end{smallmatrix} \right) = \langle \left( \begin{smallmatrix}
				0 \\ 1 \\ 1 \\ 0
			\end{smallmatrix} \right) \rangle\text{.}
		\end{equation*}
		Wir erhalten somit
		\begin{equation*}
			S = \begin{pmatrix}
				0 & 0 & -1 & 0 \\
				1 & 0 & -1 & 1 \\
				0 & 0 & -1 & 1 \\
				0 & -1 & 0 & 0 
			\end{pmatrix}\text{.}
		\end{equation*}
\end{enumerate}

\newpage

%----------------------------------------------------------------------------------------
%	FRÜHJAHR 2014
%----------------------------------------------------------------------------------------
\section{Frühjahr 2014}

\subsection{Aufgabe}
Sei \( A \in \C^{5 \times 5} \) mit \( \rk(A)= \spur(A)=3 \) und höchstens zwei Eigenwerten. Es sei \( J = \jnf(A) \).
\begin{enumerate}
	\item Wie viele Jordankästchen zum Eigenwert \( 0 \) besitzt \( J \)? 
	\item Wieso hat \( A \) einen Eigenwert \( \neq 0 \)?
	\item Welche Zahlen können als Dimension des Hauptraums zum von \( 0 \) verschiedenen Eigenwert auftreten?
	\item Bestimmen Sie alle Möglichkeiten für \( J \) unter den gegebenen Einschränkungen.
\end{enumerate}

\subsection{Ansatz}
\begin{enumerate}
	\item Benutze die Dimensionsformel.
	\item Betrachte die Spur von \( A \) und \( J \).
	\item Ermittle die Dimension des Hauptraums von \( 0 \) und beachte, dass die Summe der Haupträume \( 5 \) sein muss.
	\item Gehe die verschiedenen Möglichkeiten für die Dimensionen der Haupträume (oder der algebraischen Vielfachheiten der Eigenwerte) durch. 
\end{enumerate}

\subsection{Lösung}
\begin{enumerate}
	\item Die Anzahl an Jordankästchen ist
		\begin{align*}
			\mu_g(0) &= \dim(\eig(A,0)) = \dim\kr(A-0I_5) = \dim\kr(A) = 5 - \dim\bild(A) \\
			 &= 5 - \rk(A) = 2\text{.}
		\end{align*}
	\item Es ist \( \spur(A) = \spur(J) = 3 \). Da bei \( J \) (Dreiecksmatrix) die Spur aus den Eigenwerten besteht, muss es einen Eigenwert \( \neq 0 \) geben.
	\item Der Hauptraum zum Eigenwert \( 0 \) ist mindestens zweidimensional (weil es zwei Jordankästchen gibt), also muss der Hauptraum zum verbleibenden Eigenwert \( \lambda \) zwischen \( 1 \) und \( 3 \) Dimensionen haben.
	\item Wir gehen die verschiedenen Möglichkeiten für die algebraischen Vielfachheiten durch.
		\begin{enumerate}
			\item \underline{Fall 1}: \( \mu_a(1)=4 \). Es gibt zwei Möglichkeiten:
			\begin{equation*}
				J = \left( \begin{array}{cccc|c}
					0 & & & &  \\
					 & 0 & 0 & 0 &  \\
					 & 1 & 0 & 0 &  \\
					 & 0 & 1 & 0 &  \\
					\hline
					 & & & & 3

				\end{array} \right) \quad \text{oder} \quad J= \left( \begin{array}{cccc|c}
					0 & 0 & & &  \\
					1 & 0 & & &  \\
					 & & 0 & 0 &  \\
					 & & 1 & 0 &  \\
					\hline
					 & & & & 3 
				\end{array} \right)
			\end{equation*}
			\item \underline{Fall 2}: \( \mu_a(1)=3 \). Es gibt zwei Möglichkeiten:
			\begin{equation*}
				J = \left( \begin{array}{ccc|cc}
					0 & & & &  \\
					 & 0 & 0 & &  \\
					 & 1 & 0 & &  \\
					\hline
					 & & & \tfrac{3}{2} & 0 \\
					 & & & 1 & \tfrac{3}{2}
				\end{array} \right) \quad \text{oder} \quad J = \left( \begin{array}{ccc|cc}
					0 & & & &  \\
					 & 0 & 0 & &  \\
					 & 1 & 0 & &  \\
					\hline
					 & & & \tfrac{3}{2} &  \\
					 & & & & \tfrac{3}{2}
				\end{array} \right)
			\end{equation*}
			\item \underline{Fall 3}: \( \mu_a(1)=2 \). Es gibt drei Möglichkeiten:
			\begin{equation*}
				J = \left( \begin{array}{cc|ccc}
					0 & & & &  \\
					 & 0 & & &  \\
					\hline
					 & & 1 & 0 & 0 \\
					 & & 1 & 1 & 0 \\
					 & & 0 & 1 & 1 
				\end{array} \right), \quad J = \left( \begin{array}{cc|ccc}
					0 & & & &  \\
					 & 0 & & &  \\
					\hline
					 & & 1 &  &  \\
					 & &  & 1 & 0 \\
					 & &  & 1 & 1 
				\end{array} \right), \quad J = \left( \begin{array}{cc|ccc}
					0 & & & &  \\
					 & 0 & & &  \\
					\hline
					 & & 1 &  &  \\
					 & &  & 1 & \\
					 & &  & & 1 
				\end{array} \right)
			\end{equation*}
		\end{enumerate}
\end{enumerate}

\newpage

%----------------------------------------------------------------------------------------
%	HERBST 2014
%----------------------------------------------------------------------------------------
\section{Herbst 2014}

\subsection{Aufgabe}
Gegeben sei
\begin{equation*}
	A \coloneqq \begin{pmatrix}
		9 & -7 & 0 & 2 \\
		7 & -5 & 0 & 2 \\
		4 & -4 & 2 & 1 \\
		0 & 0 & 0 & 2 
	\end{pmatrix} \in \R^{4 \times 4}\text{.}
\end{equation*}
\begin{enumerate}
	\item Bestimmen Sie \( \widetilde{A} \coloneqq \jnf(A) \).
	\item Geben Sie \( S \in \gl_4(\R) \) an, sodass \( \widetilde{A} = S^{-1}AS \).
\end{enumerate}

\subsection{Ansatz}
\begin{enumerate}
	\item Berechne \( \cp_A(\lambda) \), dann \( \mu_a(\lambda) \) und \( \mu_g(\lambda) \) für alle \( \lambda \in \spec(A) \) und konstruiere so \( \jnf(A) \).
	\item Füge für das \( k \) große Kästchen zu \( \lambda \) Basisvektoren \( b_1,\dots \) aus \( \kr((A-\lambda I_4)^k) \) \\* (bzw. \( (A-\lambda I_4)^lb_i \), \( 0 \leq l \leq (k-1) \)) zur Basis hinzu.
\end{enumerate}

\subsection{Lösung}
\begin{enumerate}
	\item Wir berechnen
		\begin{equation*}
			\cp_A(\lambda) = (2 - \lambda)^4 \leadsto \spec(A) = \{ 2 \}, \ \mu_a(2) = 4
		\end{equation*}
		und anschließend
		\begin{equation*}
			\mu_g(2) = \dim(\eig(A,2)) = \dim(\kr(A - 2I_4)) = \dim\ker\left( \begin{smallmatrix}
				7 & -7 & 0 & 2 \\
				7 & -7 & 0 & 2 \\
				4 & -4 & 0 & 1 \\
				0 & 0 & 0 & 0
			\end{smallmatrix} \right) = \dim\langle \left( \begin{smallmatrix}
				1 \\ 1 \\ 0 \\ 0
			\end{smallmatrix} \right), \left( \begin{smallmatrix}
				0 \\ 0 \\ 1 \\ 0
			\end{smallmatrix} \right) \rangle = 2\text{.}
		\end{equation*}
		Es gibt also \( 2 \) Jordankästchen. Um das größte Jordankästchen zu ermitteln berechnen wir
		\begin{equation*}
			\dim(\kr((A - I_4)^2)) = \dim\kr\left( \begin{smallmatrix}
				0 & 0 & 0 & 0 \\
				0 & 0 & 0 & 0 \\
				0 & 0 & 0 & 0 \\
				0 & 0 & 0 & 0
			\end{smallmatrix} \right) = \dim(\R^4) = 4\text{,}
		\end{equation*}
		also ist das größte Kästchen ein \( 2 \)er-Kästchen und wir erhalten
		\begin{equation*}
			\widetilde{A} = \begin{pmatrix}
				2 & 0 & &  \\
				1 & 2 & &  \\
				 & & 2 & 0 \\
				 & & 1 & 2 
			\end{pmatrix}\text{.}
		\end{equation*}
	\item Wir wählen für die beiden \( 2 \)er-Kästchen Basisvektoren aus \( \kr((A-2I_4)^2)=\R^4 \). Da \( (A-2I_4)e_1=(A-2I_4)e_2 \) und \( (A-2I_4)e_3=0 \) ist
	\begin{equation*}
		C = \{ e_1, \ (A-2I_4)e_1, \ e_4, (A-2I_4)e_4 \}
	\end{equation*}
	eine mögliche Jordanbasis und wir erhalten
	\begin{equation*}
		S = \begin{pmatrix}
			7 & 1 & 2 & 0 \\
			7 & 0 & 2 & 0 \\
			4 & 0 & 1 & 0 \\
			0 & 0 & 0 & 1
		\end{pmatrix}\text{.}
	\end{equation*}
\end{enumerate}

\newpage

%----------------------------------------------------------------------------------------
%	FRÜHJAHR 2015
%----------------------------------------------------------------------------------------
\section{Frühjahr 2015}

\subsection{Aufgabe}
Es sei \( A \in \C^{5 \times 5} \) mit höchstens zwei verschiedenen Eigenwerten \( \lambda, \mu \in \C \) und \( \widetilde{A}=\jnf(A) \). Weiter gelte \( \rk(A)=3 \) und \( \spur(A)=6 \).
\begin{enumerate}
	\item Wie viele Jordankästchen zum Eigenwert \( \lambda = 0 \) besitzt \( \widetilde{A} \)?
	\item Wieso hat \( A \) einen Eigenwert \( \mu \neq 0 \)?
	\item Welche Zahlen können als Dimension des Hauptraums zum Eigenwert \( \mu \) auftreten?
	\item Bestimmen Sie alle Möglichkeiten für \( \widetilde{A} \) unter den gegebenen Einschränkungen. 
\end{enumerate}

\subsection{Ansatz}
\begin{enumerate}
	\item Nutze die Dimensionsformel.
	\item Betrachte die Spur von \( A \) und \( \widetilde{A} \).
	\item Ermittle die Möglichkeiten für \( \dim(\hau(A,0)) \) und beachte, dass die Summe der Dimensionen der Haupträume \( 5 \) sein muss.
	\item Gehe die möglichen Kombinationsmöglichkeiten für die Hauptraumdimensionen durch. 
\end{enumerate}

\subsection{Lösung}
\begin{enumerate}
	\item Die Anzahl an Jordankästchen ist \( \mu_g(0) \), also
		\begin{align*}
		 	\mu_g(0) &= \dim(\eig(A,0)) = \dim(\kr(A-0I_5)) = \dim(\kr(A)) = 5 - \dim(\bild(A)) \\
		 	 &= 5 - \rk(A) = 2\text{.}
		 \end{align*}
	\item Es ist \( \spur(A) = \spur(\widetilde{A}) \). Da auf der Diagonalen von \( \widetilde{A} \) die Eigenwerte stehen muss es einen weiteren Eigenwert \( \neq 0 \) geben.
	\item Die Matrix hat zwei Eigenwerte und die Dimension des Hauptraums zu \( 0 \) ist mindestens \( 2 \), also ist die Dimension des Hauptraums zu \( \mu \) zwischen \( 1 \) und \( 3 \).
	\item Wir gehen die verschiedenen Möglichkeiten für die Dimensionen der Haupträume durch:
		\begin{enumerate}
			\item \underline{Fall 1}: \( \dim(\hau(A,0))=4 \). Es gibt zwei Möglichkeiten:
				\begin{equation*}
				 	\widetilde{A} = \left( \begin{array}{cccc|c}
				 		0 & 0 & 0 & &  \\
				 		1 & 0 & 0 & &  \\
				 		0 & 1 & 0 & &  \\
				 		 & & & 0 &  \\
				 		\hline
				 		 & & & & 6
				 	\end{array} \right), \quad \widetilde{A} = \left( \begin{array}{cccc|c}
				 		0 & 0 & & &  \\
				 		1 & 0 & & &  \\
				 		 & & 0 & 0 &  \\
				 		 & & 1 & 0 &  \\
				 		\hline
				 		 & & & & 6
				 	\end{array} \right)
				 \end{equation*} 
			\item \underline{Fall 2}: \( \dim(\hau(A,0)) = 3 \). Es gibt zwei Möglichkeiten:
				\begin{equation*}
				 	\widetilde{A} = \left( \begin{array}{ccc|cc}
				 		0 & 0 & & &  \\
				 		1 & 0 & & &  \\
				 		 & & 0 & &  \\
				 		\hline
				 		 & & & 3 & 0 \\
				 		 & & & 1 & 3
				 	\end{array} \right), \quad \widetilde{A} = \left( \begin{array}{ccc|cc}
				 		0 & 0 & & &  \\
				 		1 & 0 & & &  \\
				 		 & & 0 & &  \\
				 		\hline
				 		 & & & 3 & \\
				 		 & & & & 3
				 	\end{array} \right)
				 \end{equation*} 
			\item \underline{Fall 3}: \( \dim(\hau(A,0)) = 2 \). Es gibt drei Möglichkeiten:
				\begin{equation*}
				 	\widetilde{A} = \left( \begin{array}{cc|ccc}
				 		0 & & & &  \\
				 	     & 0 & & &  \\
				 		\hline
				 		 & & 2 & 0 & 0 \\
				 		 & & 1 & 2 & 0 \\
				 		 & & 0 & 1 & 2
				 	\end{array} \right), \quad \widetilde{A} = \left( \begin{array}{cc|ccc}
				 		0 & & & &  \\
				 	     & 0 & & &  \\
				 		\hline
				 		 & & 2 & 0 & \\
				 		 & & 1 & 2 & \\
				 		 & & & & 2
				 	\end{array} \right), \quad \widetilde{A} = \left( \begin{array}{cc|ccc}
				 		0 & & & &  \\
				 	     & 0 & & &  \\
				 		\hline
				 		 & & 2 & & \\
				 		 & & & 2 & \\
				 		 & & & & 2
				 	\end{array} \right)\text{.}
				 \end{equation*} 
		\end{enumerate}
\end{enumerate}

\newpage

%----------------------------------------------------------------------------------------
%	HERBST 2015
%----------------------------------------------------------------------------------------
\section{Herbst 2015}

\subsection{Aufgabe}
Es sei \( \Phi \) ein Endomorphismus eines fünfdimensionalen komplexen Vektorraums \( V \) mit 
\begin{equation*}
	\cp_\Phi(X)=X^5-9X^3\text{.}
\end{equation*}
Ferner sei \( \dim(\kr(\Phi))=1 \). Bestimmen Sie \( \jnf(\Phi \circ \Phi) \).

\subsection{Ansatz}
Ermittle \( \jnf(\Phi) \) und nutze \( (\jnf(\Phi))^2 \) als Abbildungsmatrix von \( \Phi^2 \). Bestimme anschließend \( \jnf(\Phi \circ \Phi) = \jnf((\jnf(\Phi))^2) \).

\subsection{Lösung}
Wir formen um:
\begin{equation*}
	\cp_\Phi(X) = X^5-9X^3 = X^3(X-3)(X+3) \leadsto \spec(\Phi) = \{ 0, 3, -3 \}, \ \mu_a(0)=3, \ \mu_a(\pm 3) = \mu_g(\pm 3) = 1
\end{equation*}
Wir ermitteln die Anzahl an Jordankästchen zum Eigenwert \( 0 \):
\begin{equation*}
	\mu_g(0) = \dim(\eig(\Phi,0)) = \dim(\kr(\Phi-0I_5)) = \dim\kr(\Phi) = 1
\end{equation*}
und somit
\begin{equation*}
	A \coloneqq \jnf(\Phi) = \begin{pmatrix}
		3 & & & &  \\
		 & -3 & & &  \\
		 & & 0 & 0 & 0 \\
		 & & 1 & 0 & 0 \\
		 & & 0 & 1 & 0 
	\end{pmatrix}\text{,}
\end{equation*}
also hat \( \Phi^2 \) die Abbildungsmatrix \( A^2 = \left( \begin{smallmatrix}
	9 & 0 & 0 & 0 & 0 \\
	0 & 9 & 0 & 0 & 0 \\
	0 & 0 & 0 & 0 & 0 \\
	0 & 0 & 0 & 0 & 0 \\
	0 & 0 & 1 & 0 & 0 
\end{smallmatrix} \right) \). Wir ermitteln wie oben \( \cp_{A^2}(X) = (X-9)^2X^3 \) und anschließend \( \mu_g(9) = \dim(\kr(A^2-9I_5))=2 \), \( \mu_g(0) = \dim(\kr(A^2)) = 2 \). Somit ist
\begin{equation*}
	\jnf(\Phi \circ \Phi) = \left( \begin{array}{cc|ccc}
		9 & 0 &   &   &   \\
		1 & 9 &   &   &   \\
		\hline
		  &   & 0 &   &   \\
		  &   &   & 0 & 0 \\
		  &   &   & 1 & 0 
	\end{array} \right)
\end{equation*}